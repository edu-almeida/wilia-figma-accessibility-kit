%%%%%%%%%%%%%%%%%%%%%%%%%%%%%%%%%%%%%%%%%
% Modelo trabalho de conclusão de curso
% Bacharelado em Sistemas de Informação - IFNMG - Campus Salinas
% Version 2 (1º semestre 2021)
%
% Modelo adaptado de:
% Frits Wenneker (http://www.howtotex.com)
%
% License:
% CC BY-NC-SA 3.0 (http://creativecommons.org/licenses/by-nc-sa/3.0/)
%
% Texto adaptado de: http://www.fea.usp.br/media/fck/File/Roteiro_para_projeto_pesquisa.pdf
%%%%%%%%%%%%%%%%%%%%%%%%%%%%%%%%%%%%%%%%%


\documentclass[
	% -- opções da classe memoir --
	12pt,				% tamanho da fonte
	%openright,			% capítulos começam em pág ímpar (insere página vazia caso preciso)
	twoside,			% para impressão em recto e verso. Oposto a oneside
	a4paper,			% tamanho do papel.
	% -- opções da classe abntex2 --
	chapter=TITLE,		% títulos de capítulos convertidos em letras maiúsculas
	section=TITLE,		% títulos de seções convertidos em letras maiúsculas
	subsection=TITLE,	% títulos de subseções convertidos em letras maiúsculas
	subsubsection=TITLE,% títulos de subsubseções convertidos em letras maiúsculas
	% -- opções do pacote babel --
	english,			% idioma adicional para hifenização
	french,				% idioma adicional para hifenização
	spanish,			% idioma adicional para hifenização
	brazil,				% o último idioma é o principal do documento
	]{abntex2}

% ---------------------------------
% PACOTES
% ---------------------------------
%\renewcommand{\thesection}{\arabic{section}}
\usepackage{lmodern}			% Usa a fonte Latin Modern
\usepackage[T1]{fontenc}		% Selecao de codigos de fonte.
\usepackage[utf8]{inputenc}		% Codificacao do documento (conversão automática dos acentos)
\usepackage{indentfirst}		% Indenta o primeiro parágrafo de cada seção.
\usepackage{color}				% Controle das cores
\usepackage{lastpage}
\usepackage{graphicx}			% Inclusão de gráficos
\usepackage{microtype} 			% para melhorias de justificação
\usepackage{multirow}
\usepackage{amsmath,amsfonts,amssymb}
\usepackage[table]{xcolor}
\usepackage{multirow}           % Merge rows and columns in tables
\usepackage{longtable}          % Large tables
\usepackage{fancyvrb}           % Codes and Outputs commands
\usepackage{listings}           % Codes and Outputs commands
\usepackage{pdflscape}
\usepackage{float}
\usepackage{subfig}             % Subfigures
\usepackage{latexsym}           % Extra symbols of Latex
\usepackage{booktabs}
\usepackage{enumerate}
\usepackage{acronym}
\usepackage[portuguese, ruled, linesnumbered]{algorithm2e}
\usepackage[noend]{algpseudocode}
\usepackage{url}
\let\realurl\url
\renewcommand{\url}[1]{\realurl{#1}\wlog{URLX #1}}
\usepackage{lipsum}				% To generate dummy text
\usepackage{adjustbox}
\documentclass{article}
\usepackage[utf8]{inputenc}
\usepackage[brazil]{babel}
\usepackage{geometry}
\usepackage{array}
\usepackage{longtable}
\usepackage{booktabs}
\usepackage{caption}

% ---------------------------------
% Pacotes de citações
% ---------------------------------
\usepackage[brazilian,hyperpageref]{backref}% Paginas com as citações na bibl
\usepackage[alf,bibjustif]{abntex2cite}	% Citações padrão ABNT

% ---------------------------------
% Configurações do pacote backref
% Usado sem a opção hyperpageref de backref
\renewcommand{\backrefpagesname}{Citado na(s) página(s):~}
% Texto padrão antes do número das páginas
\renewcommand{\backref}{}
% Define os textos da citação
\renewcommand*{\backrefalt}[4]{
	\ifcase #1 %
		Nenhuma citação no texto.%
	\or
		Citado na página #2.%
	\else
		Citado #1 vezes nas páginas #2.%
	\fi}%
% ---------------------------------
% Configurações de aparência do PDF final
% ---------------------------------
% alterando o aspecto da cor azul
\definecolor{blue}{RGB}{41,5,195}

% informações do PDF
\makeatletter
\hypersetup{
     	%pagebackref=true,
		pdftitle={\@title},
		pdfauthor={\@author},
    	pdfsubject={\imprimirpreambulo},
	    pdfcreator={LaTeX with abnTeX2},
		pdfkeywords={abnt}{latex}{abntex}{abntex2}{projeto de pesquisa},
		colorlinks=true,       		% false: boxed links; true: colored links
    	linkcolor=blue,          	% color of internal links
    	citecolor=blue,        		% color of links to bibliography
    	filecolor=magenta,      		% color of file links
		urlcolor=blue,
		bookmarksdepth=4
}
\makeatother
% ---------------------------------
% Espaçamentos entre linhas e parágrafos
% ---------------------------------
% O tamanho do parágrafo é dado por:
\setlength{\parindent}{1.3cm}
% Controle do espaçamento entre um parágrafo e outro:
\setlength{\parskip}{0.2cm}  % tente também \onelineskip


%-----------------------------------
% Front pages
%-----------------------------------
\titulo{Wilia: Um UI kit de anotações para Documentação de acessibilidade, para leitores de tela, em projetos Web no Figma
}
\autor{Eduardo Pereira de Almeida}
\local{Salinas - Minas Gerais}
\date{\normalsize\today} % Gera automaticamente a data
\orientador{Arthur Faria Porto}
% \coorientador{Coorientador}
\instituicao{%
 	Instituto Federal do Norte de Minas Gerais - Campus Salinas
 	\par  
 	Bacharelado em Sistemas de Informação
}

\preambulo{Trabalho de conclusão de curso apresentado ao Curso de Bacharelado em Sistemas de Informação do Instituto Federal do Norte de Minas Gerais, como exigência para obtenção do grau de Bacharel em Sistemas de Informação.}

%-----------------------------------
% Set the final PDF
%-----------------------------------
\definecolor{blue}{RGB}{41,5,195} % setting the blue color

% PDF informations
\makeatletter                   
\hypersetup{
     	%pagebackref=true,
		pdftitle={\@title}, 
		pdfauthor={\@author},
    	pdfsubject={\imprimirpreambulo},
	    pdfcreator={LaTeX with abnTeX2},
		pdfkeywords={abnt}{latex}{abntex}{abntex2}{trabalho acadêmico}, 
		colorlinks=true,        % false: boxed links; true: colored links
    	linkcolor=blue,         % color of internal links
    	citecolor=blue,        	% color of links to bibliography
    	filecolor=magenta,      % color of file links
		urlcolor=blue,
		bookmarksdepth=4
}
\makeatother

%-----------------------------------
% Spacing between lines and paragraphs
%-----------------------------------
\setlength{\parindent}{1.5cm}   % Set length of indent
\setlength{\parskip}{0.2cm}     % Set length among the paragraphs 

%Index
\makeindex

%-----------------------------------
% Main document
%-----------------------------------
\begin{document}

%-----------------------------------
% Linguagem
\selectlanguage{brazil} % Texts in pt-br
\frenchspacing          % Remove obsolete spaces between phrases

%-----------------------------------
% Elementos pre-textuais
\pretextual

%-----------------------------------
% Capa
\imprimircapa                   

% Folha de rosto
\imprimirfolhaderosto

\newpage

%-----------------------------------
% Dedicatória (opcional)
\begin{dedicatoria}
    \vspace*{\fill}
    \centering
    \noindent

    \textit{À minha mãe, Maria Silvia Pereira de Almeida, \\
    e às minhas irmãs, Edna Pereira de Almeida e Edilaine Pereira Costa. \\
    Vocês foram meu pilar e a base que me manteve firme.}

    \vspace{1.5cm}

    \textit{Ao meu pai, Ubaldo de Almeida Costa (in memoriam), \\
    que infelizmente não pôde ver esta conclusão, \\
    mas me acompanha no caminho de sucesso que venho trilhando.}

    \vspace{1cm}

    \textit{Costuma-se dizer que a acessibilidade \\
    só se torna importante quando precisamos dela. \\
    Dedico este trabalho a todos que precisam dela todos os dias, \\
    como pessoas cegas, com baixa visão ou qualquer outra deficiência.}
    
    \vspace*{\fill}
\end{dedicatoria}

%-----------------------------------
% Agradecimentos (opcional)
\begin{agradecimentos}
   A conclusão deste trabalho representa o fim de uma longa jornada iniciada em 2018. Um caminho marcado por desafios que foram muito além da academia: enfrentei a pandemia, momentos de ansiedade e angústia, as cobranças da vida, a dor de perder alguém e, em muitas ocasiões, o desânimo e preocupação financeira me fez querer desistir. Por isso, este agradecimento é, antes de tudo, um reconhecimento a quem tornou esta conquista possível.

    Agradeço imensamente ao meu orientador, professor \textbf{Arthur Faria Porto}, por acreditar neste projeto quando eu já estava desistindo do curso. Sua disposição em assumir este TCC, sua motivação, elogios e melhorias à escrita foram fundamentais para que eu mantivesse a vontade. Sua orientação foi essencial para provar a importância de um tema tão relevante e ainda tão pouco tratado.
    
    Agradeço aos membros da banca examinadora, professora \textbf{Danielle Miranda Rodrigues} e professor \textbf{Leonardo Humberto Guimaraes Silva}, por aceitarem o convite, pelo tempo dedicado e pelas valiosas contribuições que certamente enriqueceram esta pesquisa.
    
    Ao \textbf{Instituto Federal do Norte de Minas Gerais}, pela oportunidade de formação.
    
    O pilar mais importante desta trajetória foi, sem dúvida, minha família.
    
    À minha mãe, \textbf{Maira Silvia Pereira de Almeida}, que sempre me apoiou, incentivou e se manteve firme por mim, mesmo quando eu não conseguia me manter. Seu apoio incondicional foi minha base para não desistir.
    
    Às minhas irmãs, \textbf{Edna Pereira de Almeida} e \textbf{Edilaine Pereira Costa}, por todo o apoio (com um agradecimento especial à Edna por todo o suporte estrutural e cuidados), carinho e até mesmo pelas cobranças, que, à sua maneira, nunca me deixaram esquecer o objetivo final.
    
    A todos que, de alguma forma, acreditaram em mim, meu muito obrigado.
    
\end{agradecimentos}

%-----------------------------------
% Epígrafe (opcional)
\begin{epigrafe}
    \vspace*{\fill}
	\begin{flushright}
		\textit{
            “O espírito humano precisa prevalecer sobre a tecnologia.”
            \vspace{0.2cm} \\
            --- Albert Einstein
        }
	\end{flushright}
\end{epigrafe}

%-----------------------------------
% Resumo (em português)
\setlength{\absparsep}{18pt} % set spacing between the paragraphs

\begin{resumo}
    A acessibilidade digital é fundamental para garantir que pessoas com deficiência possam interagir com interfaces de forma igualitária. Contudo, a ausência de especificações claras e padronizadas de requisitos de acessibilidade, especialmente relacionados a tecnologias assistivas como leitores de tela, representa um grande desafio no processo de design. Este trabalho propõe o desenvolvimento do Wilia, um \textit{UI Kit} para Figma voltado à documentação de requisitos de acessibilidade, com foco na experiência de usuários que utilizam leitores de tela. Baseado nas diretrizes WCAG (\textit{Web Content Accessibility Guidelines}) e WAI-ARIA (\textit{Web Accessibility Initiative - Accessible Rich Internet Applications}), o kit organiza componentes e guias visuais em uma estrutura modular e componentizada, integrando anotações e documentação técnica diretamente no ambiente do Figma. O resultado é uma ferramenta que facilita a comunicação entre equipes multidisciplinares, reduzindo erros de implementação semântica e operacional. Além disso, o \textit{Wilia} tem o potencial de contribuir para a criação de interfaces mais inclusivas, alinhadas às boas práticas internacionais e às políticas nacionais de acessibilidade digital.
    
    \vspace{1em}
    \textit{Palavras-chave: Acessibilidade, UI Kit, Figma, WCAG, tecnologias assistivas, componentes reutilizáveis, atributos semânticos, design inclusivo, código acessível.}
\end{resumo}

%-----------------------------------
% Abstract (in English) (opcional)
\begin{resumo}[Abstract]
    \begin{otherlanguage*}{english}
        Digital accessibility is essential to ensure that people with disabilities can interact with digital interfaces on an equal basis. However, the absence of clear and standardized specifications for accessibility requirements, especially related to assistive technologies like screen readers, represents a significant challenge in the design process. This paper proposes the development of Wilia, a \textit{UI Kit} for Figma focused on documenting accessibility requirements, specifically for the screen reader user experience. Based on the WCAG (\textit{Web Content Accessibility Guidelines}) and WAI-ARIA (\textit{Web Accessibility Initiative - Accessible Rich Internet Applications}) guidelines, the kit organizes components and visual guides in a **modular and component-based structure**, integrating annotations and technical documentation directly within the Figma environment. The result is a tool that facilitates communication between multidisciplinary teams, reducing semantic and operational implementation errors. Furthermore, \textit{Wilia} has the potential to contribute to the creation of more inclusive interfaces, aligned with international best practices and national digital accessibility policies.

        \vspace{1em}
        
        \textit{Keywords: Accessibility, UI Kit, Figma, WCAG, assistive technologies, reusable components, semantic attributes, inclusive design, accessible code.}
    \end{otherlanguage*}
\end{resumo}

%-----------------------------------
% Lista de figuras
\pdfbookmark[0]{\listfigurename}{lof}
\listoffigures*
\cleardoublepage

%-----------------------------------
% Lista de tabelas
\pdfbookmark[0]{\listtablename}{lot}
\listoftables*
\cleardoublepage

%-----------------------------------
% Abreviações (opcional)
%-----------------------------------
% Abbreviations and acronyms
%-----------------------------------

\begin{siglas}
    \item[IFNMG] Instituto Federal do Norte de Minas Gerais
    \item[Ex.] Exemplo
    \item[ONG] Organização sem fins lucrativos
    \item[WebAIM] Web Accessibility In Mind
    \item[OMS] Organização Mundial da Saúde
    \item[WHO] World Health Organization
    \item[W3C] World Wide Web Consortium
    \item[WCAG] Web Content Accessibility Guidelines
    \item[WAI-ARIA] Accessible Rich Internet Applications Suite
\end{siglas}
\cleardoublepage

%-----------------------------------
% Símbolos (opcional)
\input{pre-textuais/simbolos}
\cleardoublepage

%-----------------------------------
% Sumário
\pdfbookmark[0]{\contentsname}{toc}
\tableofcontents*
\cleardoublepage


%-----------------------------------
% TEXTUAL ELEMENTS
%-----------------------------------
\textual

%-----------------------------------
% Chapters
\chapter{Introdução} 
\label{cap1_introducao} 



A acessibilidade digital é um princípio ético e legal fundamental para garantir a inclusão de pessoas com deficiência no uso de tecnologias. No Brasil, 18,6 milhões de pessoas vivem com algum tipo de deficiência, representando 8,9\% da população. Desse percentual, 3,1\% são pessoas de 2 anos ou mais de idade com dificuldade para enxergar, mesmo usando óculos ou lentes de contato. Esse percentual fica somente atrás dos 3,7\% que representam pessoas com dificuldade para andar ou subir degraus \cite{ibge_2022}. 

Uma pesquisa realizada pela \citeonline{bigdatacorp_2024} em parceria com o Movimento Web para Todos \cite{mwpt_2024}, estampou que somente 2,9\%  dos sites brasileiros passaram em todos os testes de acessibilidade. Além disso, somente 21\% das pessoas com deficiência acreditam que os sites e aplicativos atendem suas necessidades, refletindo uma falta significativa na inclusão digital no país \cite{handtalk2023}. Segundo \citeonline{handtalk2023}, 72\% dos componentes de interface em sistemas vão em desencontro com as diretrizes do W3C que define como tornar conteúdos e interfaces webs dinâmicas acessíveis para pessoas com deficiência, para dinamicidade. Com base nisso, pode-se questionar, qual a dificuldade em manter sites e sistemas acessíveis para todos? \begin{comment}
sendo algumas hipóteses levantadas sobre a falta de ferramentas para mitigar esses problemas e também um possível déficit formativo dos profissionais da área. A falta de acessibilidade digital, pode não somente, limitar o acesso à informação, mas também impedir que pessoas com deficiência participem plenamente do ambiente digital.\end{comment}




%Esse déficit formativo explica por que o72\% dos componentes de interface em sistemas descumprem os padrões  WAI-ARIA para dinamicidade. A falta de acessibilidade digital não somente limita o acesso à informação, mas também impede que pessoas com deficiência participem plenamente do ambiente digital. 

%No cenário mundial, o relatório de 2024 sobre a acessibilidade das 1.000.000 principais páginas online \cite{webaim_million_2023}, evidenciou que 56.791.260 erros de acessibilidade distintos foram detectados, uma média de 56,8 erros por página. O número de erros detectados aumentou notavelmente (13,6\%) desde a análise de 2023, que encontrou 50 erros/página. “Erros” são barreiras de acessibilidade com impacto notável no usuário final, mostrando uma probabilidade muito alta de serem falhas de conformidade com WCAG 2.2.

% Esses cenários evidenciam que a maioria das empresas ainda enfrenta desafios na implementação efetiva das diretrizes WCAG e WAI-ARIA. Há uma notável desconexão entre design e implementação técnica, manifestada nos números negativos de sistemas e páginas onlines repercutidos nas pesquisas analisadas.
 
Outro fator que pode contribuir para a intensificação do problema é a possível ausência de ferramentas padronizadas destinadas à especificação de requisitos de acessibilidade durante a etapa de prototipação. Visto que, existe uma dificuldade de se encontrar pesquisas e ferramentas na área. Essa lacuna pode dificultar a comunicação entre designers e desenvolvedores, resultando em entregas imprecisas e aumentando a probabilidade de erros na implementação de funcionalidades acessíveis. O “Estudo sobre o Panorama da Acessibilidade
Digital no Brasil” realizado pela \citeonline{handtalk2023}, diz que menos de 1\% dos sites nacionais analisados possuem mecanismos básicos de acessibilidade, comprometendo a experiência de usuários que dependem de tecnologias assistivas e coloca em risco o cumprimento da legislação, como o Estatuto da Pessoa com Deficiência.


%A ausência de ferramentas padronizadas e eficientes que possibilitem a especificação de requisitos de acessibilidade em protótipos gera dificuldades significativas na comunicação entre designers e desenvolvedores. Essa lacuna resulta em entregas imprecisas, aumentando o risco de erros na implementação de funcionalidades acessíveis.  menos de 1\% dos sites avaliados no Brasil não atendem aos critérios mínimos de acessibilidade, impactando diretamente a experiência de usuários que utilizam tecnologias assistivas e mesmo conformidade legal como o Estatuto da Pessoa com Deficiência.





%As Diretrizes de Acessibilidade para Conteúdo Web (WCAG) estabelecem claramente que interfaces digitais devem ser perceptíveis, operáveis, compreensíveis e robustas, a fim de atender às necessidades de pessoas com diferentes tipos de deficiência \cite{w3c_diretrizes}. No entanto, as empresas enfrentam desafios na implementação efetiva dessas diretrizes, assim como das especificações WAI-ARIA. 

%Essa lacuna torna-se ainda mais crítica ao observarmos que 61\% das empresas brasileiras não oferecem treinamento sobre práticas acessíveis para suas equipes de desenvolvimento, resultando em componentes digitais que frequentemente descumprem os padrões definidos pelo Consórcio World Wide Web (W3C), conforme o estudo “O Panorama da Acessibilidade Digital” realizado pela \cite{handtalk_panorama_2023}. O estudo também aponta que, embora os profissionais de design demonstrem maior sensibilidade ao tema da acessibilidade, os desenvolvedores enfrentam dificuldades concretas para transformar os protótipos acessíveis em soluções técnicas adequadas, devido à falta de capacitação, documentação clara e integração entre áreas.Essa constatação é reforçada por estudo da WebAIM, que identificou que muitos desenvolvedores não recebem instruções claras sobre como aplicar acessibilidade técnica em seus projetos, e frequentemente não têm acesso a ferramentas que facilitem a validação de componentes acessíveis durante o desenvolvimento \cite{webaim_survey_2023}.Ao considerar o desenvolvimento de sistemas acessíveis como foco, é possível identificar uma lacuna significativa entre a etapa de design e a de desenvolvimento. A ausência de ferramentas práticas que promovam uma transição fluida entre o protótipo e o código, especialmente com foco em leitores de tela — recurso essencial para pessoas com deficiência visual —, contribui para a persistência de produtos digitais excludentes.

Neste contexto, UI Kits (\textit{User Interface Kits}), são conjuntos prontos de elementos gráficos e componentes de interface que facilitam o design e o desenvolvimento visual de aplicações, sites e sistemas. Essas ferramentas são amplamente usadas por designers de interface e desenvolvedores de interface para acelerar o processo de criação de produtos digitais. O destaque para o problema da acessibilidade é que elas contêm documentações inclusivas, representando um avanço estratégico no desenvolvimento web.  Segundo a \textit{Interaction Design Foundation}, esses kits colaboram para a consistência, agilidade de prototipagem e alinhamento entre design e código-fonte da aplicação, incluindo elementos essenciais para uma interface acessível por incorporarem componentes pré-configurados com contraste adequado, navegação por teclado e rotulagem semântica \cite{idf_ui_kits}. %Já o framework UIkit (\textit{User Interface Kits}), que são conjuntos de elementos visuais prontos (como botões, menus, formulários e ícones), tem como objetivo auxiliar designers e desenvolvedores a criar interfaces de forma mais rápida, padronizada e consistente.
%Eles demonstram como kits acessíveis podem oferecer componentes interativos que já seguem boas práticas de usabilidade e WAI-ARIA.
%A WAI-ARIA (\textit{Web Accessibility Initiative – Accessible Rich Internet Applications}) é uma especificação do W3C que define atributos para tornar conteúdos e aplicações web mais acessíveis a pessoas com deficiência, especialmente usuários de tecnologias assistivas, como leitores de tela.
Esses kits servem como exemplos técnicos eficazes para designers e desenvolvedores \cite{uikit_accessibility}.

%Integrar a acessibilidade desde o início é mais eficiente e econômico do que corrigir posteriormente, pois evita custos extras e retrabalho, conforme \cite{montana_accessibility}. Além disso, construir sites inclusivos deve abranger múltiplos tipos de deficiência (como visão e habilidade motora), e que recursos acessíveis integrados ao UI Kit facilitam essa aplicação já na fase inicial \cite{webdev_accessibility}.

%Dessa maneira, UI Kits acessíveis funcionam como uma ponte técnica entre designers e desenvolvedores por padronizarem componentes, documentarem requisitos de acessibilidade (como WAI‑ARIA) e reduzirem incertezas no \textit{hand-off}, promovendo sistemas web verdadeiramente inclusivos sem prejudicar a produtividade. A proposta deste trabalho alinha-se diretamente a essa necessidade, ao tornar simples e prática a documentação técnica para leitores de tela desde o design inicial.

%Em pesquisa conduzida pela \citeonline{nttdata}, foram analisadas as preferências de usuários de leitores de tela no Brasil, com base em 564 respostas válidas. Observou-se que, no desktop, a maioria utiliza o NVDA em conjunto com os navegadores Chrome ou Firefox. Já em dispositivos móveis, os leitores mais usados são o TalkBack com Chrome, o VoiceOver com Safari e o JieShuo (chinês) com Chrome. Esses dados ajudam a direcionar o desenvolvimento para os navegadores e ferramentas mais utilizados pelo público.

Assim, o presente trabalho apresenta o \textit{\textbf{Wilia}},  um Kit de Interface de Usuário para documentar especificações de acessibilidade em projetos WEB. Com o foco nos designers, a ferramenta visa promover recursos que promovem protótipos com documentação clara e tecnicamente embasada, pensando nos requisitos de acessibilidade. O presente trabalho tem foco exclusivo na acessibilidade via leitores de tela.
%A acessibilidade digital é uma exigência legal no Brasil (Lei nº 13.146/2015 - LBI) e internacional (ADA, EUA; EAA, União Europeia). No entanto, dados da pesquisa "O Panorama da Acessibilidade Digital" \cite{handtalk_panorama_2023} revelam que cerca de 61\% das empresas brasileiras não oferecem treinamento sobre práticas acessíveis e somente 21\% das pessoas com deficiência acreditam que sites e aplicativos atendem de fato suas necessidades.
%A nível global, o relatório anual WebAIM Million (2025) constatou que 94,8\% das páginas iniciais avaliadas apresentavam falhas em acessibilidade automaticamente detectáveis, com uma média de 56,8 erros por página, sendo que os mais comuns envolviam contraste insuficiente, ausência de texto alternativo e má estrutura semântica \cite{webaim_million_2025}.
%O projeto alinha-se ao Plano Nacional de Direitos da Pessoa com Deficiência \cite{pndpd_2023}, que prioriza a criação de ferramentas para democratizar a implementação de padrões acessíveis em todas as fases do desenvolvimento digital. 
Ao traduzir normas técnicas em componentes práticos, o UI Kit proposto serve como ponte entre a teoria da acessibilidade e sua aplicação concreta, contribuindo para a redução da dependência de desenvolvedores em relação a documentações vagas e a garantir a conformidade legal e ética, melhorando a experiência de usuários com deficiência.

\section{Objetivos}

\subsection{Objetivo Geral} 
Criar um UI Kit para Figma que permita documentar especificações de acessibilidade em protótipos, que busca mitigar erros de implementação semântica, descritiva e operacional. 
%observados em análises da \cite{webaim_million_2023} especificamente para leitores de tela.

\subsection{Objetivos Específicos} 
\begin{itemize}
\item  Levantar e sintetizar as diretrizes de acessibilidade da WCAG (\textit{Web Content Accessibility Guidelines}) e WAI-ARIA (\textit{Web Accessibility Initiative – Accessible Rich Internet Applications}).
\item Desenvolver um Kit de Interface de Usuário (UI Kit) na plataforma Figma.
\item Mapear componentes para documentar acessibilidade para protótipos HTML.
\item Criar material educativo: Guia prático para designers sobre a aplicação dos componentes e sua relação com requisitos técnicos.
\item Proporcionar um método claro e integrado ao fluxo de design 
%\begin{itemize}
%\item \textbf{Textos Alternativos (Alt Text)}: Componentes para rotular imagens e ícones, seguindo as recomendações da WCAG 1.1.
%\item \textbf{Hierarquia de Títulos}: Componentes para estruturar cabeçalhos (H1 a H6) e garantir navegação semântica.
%\item \textbf{Foco de Teclado}: Componentes para especificar a ordem e comportamento do foco durante a navegação por teclado (WCAG 2.4.3).
%\item \textbf{Rótulos de Formulário}: Componentes para associar rótulos a campos de entrada, evitando erros como "campo sem descrição" (WCAG 3.3.2).
%\item \textbf{Atributos ARIA}: Componentes para adicionar atributos como aria-label e aria-live em elementos interativos.
%\item \textbf{Estrutura Semântica}: Componentes para organizar conteúdo com landmarks (ex.: <nav>, <footer>) e listas.
%\end{itemize}
%\item Disponibilizar exemplos e variações de componentes: Baseados em padrões internacionais (W3C) e leis (LBI, ADA).
%\item Demonstrar a redução de erros: Comparação entre protótipos com e sem o uso do kit, focando em problemas comuns (ex.: 72% dos componentes governamentais descumprindo WAI-ARIA, conforme \cite{webaim_million_2023}).
\end{itemize}



%\section{Justificativa/Relevância}




% A acessibilidade é uma exigência legal em muitos países, incluindo o Brasil, que dispõe da Lei Nº 13.146, de 6 de julho de 2015 Lei Brasileira de Inclusão \cite{lbi_brasil_2015}). Temos as regulamentações internacionais como a ADA, Americans with Disabilities Act \cite{ada_usa_2009}, nos EUA e a EAA, European Accessibility Act \cite{eaa_europa_2019}, na União Europeia. No âmbito ético, a acessibilidade é essencial para garantir a dignidade e autonomia de usuários com deficiências. No entanto, a falta de ferramentas práticas para designers contribui para a baixa adesão às normas \cite{handtalk_panorama_2023}. Este projeto visa preencher essa lacuna, oferecendo uma solução teórica, prática e alinhada a padrões globais.

% O projeto também se alinha às recomendações do Plano Nacional de Direitos da Pessoa com Deficiência \cite{pndpd_2023}, que prioriza a criação de ferramentas para "democratizar a implementação de padrões acessíveis em todas as fases do desenvolvimento digital". Ao traduzir normas técnicas em componentes práticos, o UI Kit proposto serve como ponte entre a teoria da acessibilidade e sua aplicação concreta no ecossistema digital.



%\section{Métodos}



 %A metodologia adotada inclui as seguintes etapas:

 %\begin{itemize}
 %    \item Pesquisa Bibliográfica: Análise das diretrizes internacionais de acessibilidade (WCAG, WAI-ARIA, LBI, ADA e EAA), identificação dos principais desafios enfrentados por designers e desenvolvedores e analise sobre a ótica de tecnologias assistivas.
  %   \item Análise de Ferramentas Existentes: Avaliação de soluções disponíveis no mercado para identificar lacunas e oportunidades.
   %  \item Desenvolvimento do UI Kit: Criação dos componentes no Figma, com base nas diretrizes identificadas e de modo a possibilitar uma especificação que vise mitigar os principais problemas apontados pela \cite{webaim_million_2023} nos sites.
    % \item Validação: Alinhamento dos componentes com as normas internacionais e disponibilização de exemplos práticos.
    %\item Documentação: Elaboração de materiais educativos que orientem o uso correto dos componentes.

 %\end{itemize}

%%%%%%%%%%%%
%Faça nesse estilo: 

%\section{ORGANIZAÇÃO DO TEXTO}
%Este trabalho está estruturado em cinco capítulos. O presente capítulo, \textbf{Introdução}, apresenta o tema do estudo, define os objetivos gerais e específicos e descreve a organização do texto. O segundo capítulo, \textbf{Revisão da Literatura}, discute trabalhos relacionados ao tema, com destaque para a pesquisa de \citeonline{silva2024} e para o processo de visualização de dados. Também são explorados, nesse capítulo, aspectos do desenvolvimento web, tanto no front-end quanto no back-end, além da apresentação de dois exemplos de plataformas web voltadas à visualização de dados.

%O terceiro capítulo, \textbf{Metodologia}, detalha todo o processo de visualização de dados e a forma como ele foi aplicado. Abrange ainda a modelagem e criação do banco de dados, bem como o desenvolvimento das partes front-end e back-end da plataforma. O quarto capítulo, \textbf{Resultados}, apresenta os resultados obtidos, incluindo as representações gráficas dos dados coletados.

%Por fim, o quinto capítulo, \textbf{Conclusão}, traz as considerações finais sobre a plataforma desenvolvida, reflexões sobre os resultados e sugestões para trabalhos futuros.

\section{Estrutura dos capítulos}
Este trabalho está estruturado em cinco capítulos. O presente capítulo, \textbf{Introdução}, apresenta o tema do estudo, define os objetivos gerais e específicos e descreve a organização do texto. O segundo capítulo, \textbf{Referencial Teórico}, discute trabalhos relacionados ao tema e apresenta com maior profundidade os temas que norteiam essa pesquisa como acessibilidade digital, tecnologias assistivas, WCAG e WAI-ARIA, além da legislação a respeito e a ferramenta Figma.

O terceiro capítulo, \textbf{Metodologia}, detalha todo o processo de construção da pesquisa, incluindo a etapa de prototipação e organização no Figma. O quarto capítulo, \textbf{Resultados}, apresenta os resultados obtidos, incluindo o mapeamento dos aspectos gráficos no Figma e criação do material educativo.

Por fim, o quinto capítulo, \textbf{Conclusão}, traz as considerações finais sobre a pesquisa desenvolvida, reflexões sobre os resultados e sugestões para trabalhos futuros.








% No mundo digital atual, onde tecnologias como sites, aplicações e plataformas online desempenham um papel fundamental na vida cotidiana, garantir que essas interfaces sejam acessíveis a todas as pessoas é uma obrigação ética, social e legal. Globalmente, cerca de 1,3 bilhão de pessoas, ou 16\% da população mundial, vivem com algum tipo de deficiência, segundo a Organização Mundial da Saúde \cite{oms_2021}. No Brasil, esse número chega a mais de 17 milhões de pessoas, quase 8\% da população, conforme apontado por uma pesquisa realizada pela \cite{handtalk_panorama_2023}. Apesar disso, apenas 21\% dessas pessoas acreditam que sites e aplicativos contemplam suas necessidades de navegação, e menos de 1\% dos sites no país contam com serviços de acessibilidade focados em pessoas com deficiência.

% O cenário global reflete uma realidade semelhante. A Web Accessibility in Mind \cite{webaim_million_2023}, uma organização sem fins lucrativos, analisou um milhão de homepages e revelou que 98\% delas apresentam pelo menos um erro de acessibilidade significativo, como problemas de contraste de cores, falta de textos alternativos para imagens e dificuldades de navegação por teclado. Esses números evidenciam um abismo entre o conhecimento sobre acessibilidade digital e sua implementação prática.

% As Diretrizes de Acessibilidade para Conteúdo Web (WCAG), publicadas pelo Consórcio World Wide Web \cite{w3c_diretrizes}, estabelecem claramente que interfaces digitais devem ser perceptíveis, operáveis, compreensíveis e robustas para atender às necessidades de pessoas com diferentes tipos de deficiência. No entanto, muitos designers e desenvolvedores enfrentam dificuldades na implementação prática desses princípios. Um estudo conduzido por \cite{gonzalez2022} identificou que a falta de clareza na especificação de requisitos de acessibilidade durante a fase de design é uma das principais barreiras para garantir interfaces verdadeiramente inclusivas.

% Nesse contexto, surge a necessidade de soluções práticas que facilitem a especificação de requisitos técnicos de acessibilidade desde o início do processo de design. Para resolver essa problemática, este trabalho propõe desenvolver uma Biblioteca de Especificações de Acessibilidade no Figma, uma ferramenta prática voltada para designers que buscam documentar clara e tecnicamente os requisitos de acessibilidade em seus protótipos.

% A biblioteca será estruturada para oferecer aos designers uma série de componentes e guias visuais que permitem marcar e especificar requisitos de acessibilidade de forma compreensível para desenvolvedores. Entre os principais elementos incluídos, destacam-se:

% \begin{itemize}
%     \item Textos Alternativos (Alt Text) : Componentes que explicam como rotular imagens e ícones para leitores de tela.
%     \item Hierarquia de Títulos: Modelos que demonstram como organizar cabeçalhos (H1, H2, etc.) para garantir uma navegação semântica.
%     Contraste de Cores: Paletas de cores pré-definidas que atendem aos níveis de contraste recomendados pela WCAG.
%     \item Foco de Teclado: Exemplos visuais de como o foco deve se comportar durante a navegação por teclado.
%     Rótulos de Formulários: Componentes que mostram como associar rótulos a campos de entrada para garantir acessibilidade.
%     \item Atributos ARIA: Guia prático para o uso correto de atributos ARIA em componentes interativos.
%     Estrutura Semântica: Modelos que demonstram como organizar conteúdo com cabeçalhos, listas e landmarks (ex.: <header>, <main>, <footer>).
% \end{itemize}

% Esses componentes serão baseados nas diretrizes internacionais da WCAG e WAI-ARIA, garantindo conformidade com padrões consolidados e facilitando a comunicação entre equipes multidisciplinares.



% \section{Objetivos}

% Este trabalho pretende desenvolver uma ferramenta prática que auxilie designers na especificação técnica de requisitos de acessibilidade em projetos digitais. Mais especificamente, busca-se:

% \begin{itemize}
%     \item Criar uma biblioteca organizada e intuitiva que forneça componentes e guias visuais para documentar requisitos de acessibilidade.
%     \item Basear a biblioteca nas diretrizes da WCAG 2.2 e WAI-ARIA , promovendo práticas inclusivas no design de interfaces digitais.
%     \item Facilitar a comunicação entre designers e desenvolvedores, reduzindo lacunas na implementação de acessibilidade.
% \end{itemize}



% \section{Justificativa e relevância}

% A criação desta biblioteca é essencial para apoiar e promover a inclusão digital, facilitando a comunicação entre designers e desenvolvedores e tornando o desenvolvimento mais preciso. Além disso, a biblioteca abordará pontos cruciais, como:

% \begin{itemize}
%     \item Impacto Social: Promover a inclusão digital, garantindo que pessoas com deficiência tenham acesso igualitário às tecnologias digitais.
%     \item Conformidade Legal: Atender às exigências legais de acessibilidade digital estabelecidas por leis como a LBI, Lei Brasileira de Inclusão \cite{lbi_brasil_2015}, a ADA, Americans with Disabilities Act \cite{ada_usa_2009}, nos EUA e a EAA, European Accessibility Act \cite{eaa_europa_2019}, na União Europeia.
%     \item Benefícios Empresariais: Interfaces acessíveis melhoram a experiência do usuário, ampliam o alcance do público-alvo e reduzem riscos legais.
%     \item Inovação Tecnológica: Contribui para o avanço do design inclusivo, preenchendo uma lacuna no ecossistema de ferramentas de design digital.
%     \item Conscientização e Aprendizado: Além de facilitar a especificação técnica de requisitos de acessibilidade, a biblioteca pode servir como uma ferramenta educacional poderosa. Ao fornecer exemplos práticos e guias claros baseados nas diretrizes da WCAG e WAI-ARIA, possibilitando designers e desenvolvedores entenderem melhor os princípios fundamentais da acessibilidade digital. Essa abordagem prática promove a conscientização sobre a importância de projetar soluções verdadeiramente inclusivas, contribuindo para um ecossistema digital mais acessível e empático.
% \end{itemize}



% \section{Métodos}

% A metodologia adotada inclui as seguintes etapas:

% \begin{itemize}
%     \item Pesquisa Bibliográfica : Análise de diretrizes da WCAG , WAI-ARIA e outras referências sobre acessibilidade digital.
%     \item Análise de Ferramentas Existentes : Avaliação de soluções disponíveis no mercado para identificar lacunas e oportunidades.
%     \item Prototipagem no Figma : Desenvolvimento da Biblioteca de Especificações com base nas diretrizes pesquisadas.
%     \item Documentação dos Componentes : Criação de guias detalhados para explicar o uso de cada componente.
% Estrutura dos Capítulos
% \end{itemize}



% \section{Estrutura dos Capítulos}

% O trabalho está organizado da seguinte forma:

% \begin{itemize}
%     \item Capítulo 1 - Introdução : Apresentação do tema, objetivos, justificativa, métodos e estrutura do trabalho.
%     \item Capítulo 2 - Referencial Teórico : Fundamentação teórica sobre acessibilidade digital, tecnologias assistivas, WCAG e WAI-ARIA.
%     \item Capítulo 3 - Metodologia : Detalhamento dos métodos utilizados no desenvolvimento da Biblioteca de Especificações.
%     \item Capítulo 4 - Resultados : Apresentação da Biblioteca finalizada e descrição de seus componentes.
%     \item Capítulo 5 - Conclusão : Síntese dos principais aprendizados e sugestões para trabalhos futuros.
%     \item Referências : Lista de todas as fontes consultadas.
% \end{itemize}

% Com essa abordagem, espera-se contribuir significativamente para a democratização do design inclusivo, impactando positivamente a experiência de milhões de usuários que dependem de interfaces acessíveis para navegar na web.






% \subsection{Objetivo Geral} 

% Estruturado no Figma, esse kit de Interface de Usuário será estruturado para oferecer aos designers uma série de componentes e guias visuais que permitem marcar e especificar requisitos de acessibilidade de forma compreensível para desenvolvedores. Entre os principais elementos incluídos, destacam-se:


% \begin{itemize}
%      \item Textos Alternativos (Alt Text) : Componentes que explicam como rotular imagens e ícones para leitores de tela.
%      \item Hierarquia de Títulos: Modelos que demonstram como organizar cabeçalhos (H1, H2, etc.) para garantir uma navegação semântica.
%      Contraste de Cores: Paletas de cores pré-definidas que atendem aos níveis de contraste recomendados pela WCAG.
%      \item Foco de Teclado: Exemplos visuais de como o foco deve se comportar durante a navegação por teclado.
%      Rótulos de Formulários: Componentes que mostram como associar rótulos a campos de entrada para garantir acessibilidade.
%      \item Atributos ARIA: Guia prático para o uso correto de atributos ARIA em componentes interativos.
%      Estrutura Semântica: Modelos que demonstram como organizar conteúdo com cabeçalhos, listas e landmarks (ex.: <header>, <main>, <footer>).
%  \end{itemize}

%  Esses componentes serão baseados nas diretrizes internacionais da \cite{wcag_22_2023}, garantindo conformidade com padrões consolidados e facilitando a comunicação entre equipes multidisciplinares. 



% \section{Justificativa/Relevância}
% Este UI Kit de Acessibilidade será essencial para apoiar e promover a inclusão digital, facilitando a comunicação entre designers e desenvolvedores e tornando o desenvolvimento mais preciso. Além disso, o Kit abordará pontos cruciais, como:

%  \begin{itemize}
%      \item Impacto Social: Promover a inclusão digital, garantindo que pessoas com deficiência tenham acesso igualitário às tecnologias digitais.
%      \item Conformidade Legal: Incentivar que projetos web sejam implementados cumprindo às exigências legais de acessibilidade digital estabelecidas por leis como a LBI, Lei Brasileira de Inclusão \cite{lbi_brasil_2015}, e até mesmo as regulações internacionais como a ADA, Americans with Disabilities Act \cite{ada_usa_2009}, nos EUA e a EAA, European Accessibility Act \cite{eaa_europa_2019}, na União Europeia.
%      \item Benefícios Empresariais: Interfaces acessíveis melhoram a experiência do usuário, ampliam o alcance do público-alvo e reduzem riscos legais.
%      \item Inovação Tecnológica: Contribui para o avanço do design inclusivo, preenchendo uma lacuna no ecossistema de ferramentas de design digital.
%      \item Conscientização e Aprendizado: Além de facilitar a especificação técnica de requisitos de acessibilidade, a biblioteca pode servir como uma ferramenta educacional poderosa. Ao fornecer exemplos práticos e guias claros baseados nas diretrizes da WCAG e WAI-ARIA, possibilitando designers e desenvolvedores entenderem melhor os princípios fundamentais da acessibilidade digital. Essa abordagem prática promove a conscientização sobre a importância de projetar soluções verdadeiramente inclusivas, contribuindo para um ecossistema digital mais acessível e empático.
%  \end{itemize}


% Este trabalho busca contribuir para reduzir essa lacuna ao fornecer uma solução prática e acessível para designers. Ao facilitar a especificação de acessibilidade desde as fases iniciais do design, espera-se que o UI Kit proposto melhore a assertividade na entrega de protótipos, otimize a comunicação entre equipes multidisciplinares e, consequentemente, promova uma experiência digital mais inclusiva.

\chapter{Revisão da Literatura} 
\label{cap2_revisao} 

A revisão da literatura aborda os tópicos em que esse trabalho está baseado, como acessibilidade na web e exclusão digital, bem como as legislações e normalizações pertinentes ao tópico.

%leitores de tela e a forma como eles fazem a interpretação das interfaces, diretrizes e padrões como WCAG, WAI-ARIA, design de interfaces e documentação no Figma e por fim, o valor de kits e anotações visuais para acessibilidade.

\section{Acessibilidade na web e exclusão digital}

Nos últimos anos, muito se tem discutido sobre inclusão digital (\cite{nttdata}, \cite{webaim_survey_2023}, \cite{abnt17225}), mas ainda são poucas as iniciativas que realmente se preocupam com a experiência de quem enfrenta dificuldades para acessar o conteúdo online. A acessibilidade digital, nesse contexto, deixou de ser um detalhe técnico e passou a ser parte essencial de um debate mais amplo sobre direitos e igualdade. Conforme \cite{filgueiras2024inclusao}, a cidadania no século XXI exige, entre outras coisas, a possibilidade de se conectar, interagir e usufruir das plataformas digitais de maneira autônoma e segura. No entanto, para uma parcela significativa da população, como por exemplo pessoas com deficiência visual, o uso da internet continua com diversas barreiras e dificuldades \cite{nttdata}. Isso mostra que, embora muitos serviços tenham migrado para o meio digital, nem todos puderam acompanhar esse movimento em condições de igualdade, tornando a exclusão digital uma realidade marcante  para pessoas com deficiência e revelando um cenário desafiador em relação à acessibilidade na web.

%Um exemplo disso pode ser visualizado na pesquisa \citeonline{cetic2024ti2c}, realizada pelo Centro Regional de Estudos para o Desenvolvimento da Sociedade da Informação (Cetic.br), que mostra que em 2023, 156 milhões de pessoas se conectaram à internet no Brasil. Por outro lado, existe uma variação expressiva entre a conectividade conforme a renda, em que o acesso à internet entre as classes D e E chega a cerca de 60\%, e entre as pessoas que recebem menos de 1 salário mínimo chega a 24\%, com a conexão compartilhada entre vizinhos, por exemplo (Centro Regional de Estudos para o Desenvolvimento da Sociedade da Informação, Cetic.br).

%Com relação as pessoas com algum tipo de deficiência, a \cite{oms_2021} cita que cerca de 16\% da população mundial vive com algum tipo de deficiência, o que corresponde a cerca de 1,3 bilhão de pessoas.
Segundo dados do Instituto Brasileiro de Geografia e Estatística \citeonline{ibge_2022}, existem mais de 6,5 milhões de pessoas com deficiência visual no Brasil, sendo 500 mil cegas e cerca de 6 milhões com baixa visão. Esses dados demonstram a importância de considerar a acessibilidade digital como uma questão de política pública e de direitos humanos.

Entretanto, o cenário atual parece não levar esses dados em consideração, conforme relatório \citeonline{webaim_survey_2023}, que avaliou a acessibilidade de um milhão de páginas da web e concluiu que 96,3\% delas apresentavam falhas básicas no atendimento das diretrizes WCAG 2.1 (\textit{Web Content Accessibility Guidelines}) . %Entre os principais erros pode-se citar a ausência de texto alternativo em imagens (tag alt), baixo contraste entre texto e fundo e a falta de etiquetas em formulários (tag label). Elementos que são básicos e que comprometem a experiência de navegação de usuários com deficiência, principalmente os que utilizam leitores de tela.

No Brasil, um estudo feito em 2024 pela BigDataCorp em parceria com o Movimento Web para Todos (MWPT) analisou mais de 21 milhões de sites brasileiros e revelou que menos de 3\% cumpriam de forma mínima os requisitos de acessibilidade digital, conforme \cite{freire2024}. %No setor público, esse descumprimento é ainda maior, onde 89,46\% dos sites governamentais possuem problemas que comprometem o uso por pessoas com deficiência, aumentando à exclusão não somente por quem não tem acesso a internet, mas também por pessoas com algum nível de deficiência.

Outra pesquisa sobre o tema foi realizada pelo de \citeonline{nttdata} demonstra que a maior deficiência pelos respondentes da pesquisa é a cegueira, e com isso também indica que, entre os sites inacessíveis para pessoas com deficiência, os elementos mais encontrados nos sites são: 
\begin{itemize}
    \item Imagens CAPTCHA apresentando texto usado para verificar que você não é um robô;
    \item Imagens sem descrição ou com descrições inadequadas;
    \item Campo de busca inacessível ou inexistente;
    \item Formulários complexos;
    \item Links ou botões que não fazem sentido;
    \item Telas ou partes de telas que mudam inesperadamente;
    \item Falta de cabeçalhos;
    \item Falta de links “ir para conteúdo” ou “ir para o menu”;
    \item Muitos elementos focados com Tab na mesma página e tabelas complexas.
\end{itemize}

Todos esses itens demonstram, que, de certa forma, que muitos sites são criados muitas vezes visando sua identidade visual, e não na acessibilidade do mesmo, problemas esses que em geral, poderiam ser resolvidos pelas etiquetas Figma propostas no trabalho.

Essa falta de acessibilidade impacta diretamente a qualidade de vida das pessoas com deficiência, pois conforme estudo realizado por \citeonline{webaim_survey_2023}, revelou-se que 71\% dos usuários com deficiência abandonam sites em que enfrentam dificuldades de acesso, o que não apenas restringe seu acesso a informação como também compromete sua participação no mercado de trabalho, na educação e no consumo. Em um contexto onde serviços básicos como agendamento médico, matrícula escolar, serviços bancários e pagamentos de impostos são feitos hoje em dia online, a exclusão digital representa uma forma moderna de segregação.

\subsection{Legislação}
Do ponto de vista legal, o Brasil possui legislação específica referente a acessibilidade digital na \citeonline{brasil2015}, ou seja, Lei Brasileira de Inclusão da Pessoa com Deficiência (Estatuto da Pessoa com Deficiência), que estabelece em seu artigo 63, que os sites mantidos por empresas com sede ou representação no país devem assegurar acessibilidade para pessoas com deficiência, em conformidade com os padrões estabelecidos para a web. Tem-se ainda o o Modelo de Acessibilidade em Governo Eletrônico (eMAG) que foi lançado em 2007, com alinhamento as diretrizes internacionais da W3C e que deveria ser adotado por todos os órgãos da administração pública federal \citeonline{emag}.

O Marco Civil da Internet, através da Lei nº 12.965/2014, também estabelece direitos fundamentais relacionados ao acesso digital, embora não seja esse diretamente seu foco. Assim, em seu artigo 3 ele inclui o acesso universal à internet como princípio, o que pressupõe que o acesso seja possível para todos, o que incluindo pessoas com deficiência; em seu artigo 7 ele trata dos direitos dos usuários, destacando o direito à acessibilidade da informação, à neutralidade da rede e à inviolabilidade da privacidade, além de incentivar políticas públicas que promovam a inclusão digital, incluindo pessoas com deficiência \cite{marcoCivil}.

Outra legislação brasileira que promove a acessibilidade na web é a \citeonline{abnt17225}, uma norma brasileira publicada pela Associação Brasileira de Normas Técnicas (ABNT) que trata especificamente da acessibilidade na internet. Ela foi criada para estabelecer diretrizes e critérios técnicos com o objetivo de garantir que websites, portais, sistemas e conteúdos digitais sejam acessíveis a todas as pessoas, incluindo aquelas com deficiência. Como norte a mesma tem a base fundamentada nas WCAG 2.1, adaptadas para o contexto brasileiro, e aplica-se a portais públicos e privados, aplicativos móveis, plataformas educacionais, e-commerces, entre outros. Além disso, a mesma aborda elementos como o texto alternativo para imagens, navegação via teclado, contraste de cores, fontes legíveis, uso de leitores de tela para navegação e formulários acessíveis \citeonline{wcag_22_2023}.


\section{Diretrizes e Padrões}
A acessibilidade na web é fundamentada em diretrizes técnicas desenvolvidas por organizações internacionais, especialmente pelo W3C (\textit{World Wide Web Consortium}. Entre essas, destacam-se as \textit{Web Content Accessibility Guidelines} (WCAG) e as especificações do \textit{Accessible Rich Internet Applications} (WAI-ARIA), ambas desenvolvidas pelo grupo \textit{Web Accessibility Initiative} (WAI).

\subsection{WCAG}
Já no cenário internacional, as WCAG vêm sendo atualizadas de forma regular para englobar novos desafios, por exemplo, em sua versão 2.1 atualizada em Maio de 2025, adicionou-se critérios voltados a dispositivos móveis e à acessibilidade cognitiva \citeonline{wcag_22_2023}. Apesar disso e da boa intenção dessas técnicas, a aplicação prática de fato dessas diretrizes ainda é limitada pela falta e pela dificuldade de fiscalização, desconhecimento técnico por parte dos desenvolvedores, pressão para desenvolvimento dos produtos de forma rápida, o que faz muitas vezes com que princípios básicos de acessibilidade sejam ignorados e ausência de políticas de capacitação.

As WCAG têm como objetivo tornar o conteúdo web mais acessível para pessoas com deficiências, incluindo deficiências visuais, auditivas, motoras e cognitivas. Atualmente ela esta na versão 2.2 (2023) e sua evolução histórica pode ser visualizada na Figura \ref{img_historico}.

\begin{figure}[H] 
    \centering
    \caption{Evolução histórica do WCAG.}
    \includegraphics[width=1.0\textwidth]{Figuras/historico.PNG}
    \label{img_historico}
    \caption{Fonte: \cite{articleHistorico}}
  \end{figure}

Conforme \cite{articleHistorico}, cada nova versão tem um foco e uma evolução da anterior, sendo que a versão 1.1 de 1999 tinha como objetivo as recomendações técnicas, porém era pouco flexível, o que trazia dificuldades na sua aplicação a novos formatos e tecnologias, mesmo assim, a versão 2.0 levou quase 10 anos para ser lançada, em 2008, e já trouxe uma abordagem mais tecnológica e independente de linguagem, permitindo sua aplicação em diferentes contextos como em HTML, PDF, aplicativo, entre outros. A versão de 2018, 2.1, introduziu novos critérios voltados à acessibilidade mobile, considerando deficiências cognitivas, baixa visão e acessibilidade por toque. Por fim, a versão 2.2 lançada em 2023 acrescentou novos critérios relacionados a controles visíveis, navegação com teclado e foco, buscando melhorar a experiência de navegação para usuários com dificuldades motoras e cognitivas \cite{wcag_22_2023}.

Alguns autores, como \cite{articleHistorico} tem se debruçado sobre a análise da evolução das diretrizes de acessibilidade na web, e destacam a crescente importância da conformidade para a melhoria da experiência do usuário e para a criação de ambientes digitais mais inclusivos. Dessa forma, os autores analisam os contextos históricos dos principais marcos no desenvolvimento das normas de acessibilidade, e através disso, investigam os impactos da acessibilidade indo além do cumprimento legal, e ressaltam seu papel na construção de experiências digitais acessíveis e centradas no usuário, independente de qualquer deficiência que venham a ter.

Inclusive o trabalho de \cite{shah2023wcag} abordou o processo de atualização da versão WCAG 2.0 para WCAG 2.1, enfatizando a necessidade de manter-se atualizado com as novas versões das normas e com as novas tecnologias, como a mobile. Com base nisso a pesquisa propõe uma estratégia para atualização para a nova versão dividida em 4 etapas principais, sendo: avaliação, planejamento estratégico, implementação e testes, e com base nisso, as empresas conseguem mensurar de forma mais assertiva o esforço para ajustar seus sites/ sistemas. Ainda é ressaltado na pesquisa a importância das equipes interdisciplinares e do envolvimento de usuários no processo, bem como as barreiras enfrentadas na migração e os aprendizados obtidos, oferecendo uma visão prática e realista dos desafios.

Já o trabalho de \cite{zdravkova2022remote} avalia a acessibilidade em sistemas de gestão de aprendizagem, aplicativos de videoconferência com áudio e vídeo, e cursos online abertos massivos (MOOCs) com base em 4 deficiências, sendo: motora, visual, auditiva e cognitiva com base nas recomendações da WCAG 2.1. Tal avaliação foi feita durante o período do covid-19, em que muitas instituições de ensino substituíram o ensino presencial pelo virtual, mostrando a fragilidade que alunos com deficiência tiveram que enfrentar para continuar estudando.

Com base nos resultados da pesquisa, são propostas recomendações para tornar o aprendizado online mais acessível a estudantes com necessidades especiais, com o objetivo de garantir uma educação ampla para todos, sem discriminação por motivo de deficiência.

Apesar das novas versões terem trazido conteúdo com aplicação em outros contextos, os princípios do WCAG continuam os mesmos que em sua primeira versão, segundo \cite{wcag_21}, conforme a sigla POUR (\textit{Perceivable - percepção, Operable - operável, Understandable - compreensível, Robust - robusto}).

Quanto a Percepção, tem-se que a informação e os componentes da interface devem ser apresentados de forma que possam ser percebidos por todos os usuários, independentemente de qualquer deficiência, o que inclui diferentes formas de navegação e acesso ao conteúdo, por exemplo. Relacionado ao princípio Operável, seu objetivo é que os componentes da interface e a navegação devem estar disponíveis e funcionais por meio de diferentes formas de interação, como teclado ou comandos de voz. O princípio Compreensível dita que o conteúdo deve ser claro e previsível, permitindo fácil entendimento, independente do nível de conhecimento do usuário quando ao uso daquela ferramenta, e por fim, o item Robusto preza pela compatibilidade do conteúdo entre diferentes agentes de usuário, o que engloba as tecnologias assistivas.

Além disso, o WCAG trabalha com 3 níveis de conformidade, A, AA e AAA, e elas são utilizadas para mensurar o quanto um site atende aos requisitos de acessibilidade. Cada nível representa um conjunto de critérios de sucesso com exigências crescentes \cite{wcag_22_2023}.

O nível A é considerado o mínimo básico de conformidade, atendendo aos requisitos básicos de acessibilidade, e para isso ele deve remover as barreiras mais graves que impedem o acesso a pessoas com deficiência, como por exemplo ter a opção de texto para os conteúdos visuais ou permitir a navegação do conteúdo via teclado.

O nível AA é o recomendado, pois fornece um melhor equilíbrio entre acessibilidade e viabilidade técnica e estética. Alguns exemplos incluem a possibilidade de redimensionar o conteúdo até 200\% sem perda de conteúdo, as páginas terem títulos descritivos com a marcação correta (h1, h2, etc) e um contraste mínimo de cor entre o texto e o fundo.

Por fim, o nível AAA é considerado avançado e como o maior nível de conformidade envolvendo a acessibilidade, exigindo o cumprimento de todos os requisitos anteriores e ainda mais requisitos como por exemplo aumentar o contraste entre texto e fundo para 4:1, oferecer a definição de palavras incomuns ou jargões, fornecer legendas para conteúdo de video e também a interface adaptada para diversos estilos cognitivos e de aprendizagem. A Tabela \ref{tab:wcag-niveis} apresenta um resumo comparativo entre os níveis.

\begin{table}[H]
\centering
\caption{Resumo comparativo dos níveis de conformidade da WCAG}
\label{tab:wcag-niveis}
\begin{tabular}{|c|c|c|}
\hline
\textbf{Nível} & \textbf{Foco Principal} & \textbf{Aplicabilidade} \\
\hline
A & Acessibilidade básica & Remove as barreiras críticas \\
\hline
AA & Acessibilidade robusta & Recomendado para a maioria dos sites \\
\hline
AAA & Acessibilidade avançada & Ideal, mas nem sempre viável ou exigido \\
\hline
\end{tabular}
\caption*{\textbf{Fonte}: Do Autor, 2025.}
\end{table}


\subsection{WAI-ARIA}
Já o WAI-ARIA, por sua vez, fornece um conjunto de atributos que podem ser adicionados ao HTML para descrever melhor os elementos da interface para tecnologias assistivas, como leitores de tela. Esses atributos são especialmente importantes para tornar aplicativos web dinâmicos mais compreensíveis por usuários com deficiência \cite{wicaria2023}.

Importante diferenciar primeiramente a semântica nativa da WAI-ARIA, assim sendo, a semântica nativa refere-se ao uso apropriado dos elementos HTML que já possuem significado implícito e comportamento esperado pelos navegadores e tecnologias assistivas, como por exemplo elementos como botões, formulários, links, entre outros, e esses são automaticamente reconhecidos por leitores de tela, permitindo que usuários com deficiência naveguem e interajam com uma página sem necessidade de configurações adicionais. Esses elementos, quando utilizados corretamente, promovem acessibilidade de forma confiável, padronizada e eficiente~\cite{mozilla_html}.

Por outro lado, a WAI-ARIA, como mencionado, é uma especificação do W3C que fornece atributos adicionais para melhorar a acessibilidade de interfaces ricas criadas com JavaScript e HTML dinâmico, permitindo a definição de papéis chamados de \textit{roles}, além de estados e propriedades que descrevem o comportamento de componentes de interface que não possuem um equivalente semântico direto em HTML, como menus personalizados, carrossel de imagens, abas ou diálogos modais, conforme \cite{w3c_aria}.

Apesar da utilidade do ARIA em casos específicos, sua aplicação deve ser feita com cuidado, pois uma das diretrizes mais importantes da acessibilidade moderna menciona o seguinte: \textit{"First Rule of ARIA: Don’t use ARIA"} \cite{wicaria2023}. Assim, a regra afirma que sempre que existir um elemento HTML com semântica nativa apropriada, ele deve ser preferido em vez do uso de ARIA, pois os elementos nativos são automaticamente compreendidos por navegadores, leitores de tela e demais tecnologias assistivas, enquanto o ARIA depende de implementação correta, o que frequentemente não acontece.

Um exemplo do uso incorreto de ARIA seria por exemplo aplicar a tag de botão  \texttt{role="button"} a uma \texttt{<div>} sem suporte ao teclado, o que poderia comprometer seriamente a acessibilidade da interface. Com base nisso, o uso do ARIA deve ser reservado apenas para situações em que a semântica nativa não é suficiente ou não existe, como no desenvolvimento de \textit{widgets} customizados, e nesse caso, seria necessário o uso dos atributos ARIA com precisão técnica, seguindo as especificações do W3C e sendo testados com tecnologias assistivas reais, como por exemplo em leitores de tela.

Exemplos de como o uso incorreto do padrão podem impactar negativamente na navegação é apresentado por \cite{articleCoreia}, em que os autores avaliaram a acessibilidade de conteúdos em 50 sites da Coreia do Sul e 50 sites internacionais, e como resultados obtiveram que muitos sites não utilizam o WAI-ARIA ou o fazem de forma incorreta. No entanto, na Coreia do Sul, observou-se um aumento gradual no uso correto da especificação.

A \autoref{tab:semantica-vs-aria} apresenta um resumo da diferença de quando usar a web semântica e ARIA com exemplos.

\begin{table}[H]
\centering
\caption{Comparação entre Semântica Nativa e WAI-ARIA}
\label{tab:semantica-vs-aria}
\begin{tabular}{|p{4cm}|p{5cm}|p{5cm}|}
\hline
\textbf{Aspecto} & \textbf{Semântica Nativa (HTML)} & \textbf{WAI-ARIA} \\
\hline
\textbf{Definição} &
Elementos HTML com significado e comportamento acessível embutidos. &
Atributos que descrevem papéis, estados e propriedades para componentes personalizados. \\
\hline
\textbf{Compatibilidade} &
Altamente compatível com navegadores e leitores de tela. &
Depende de implementação correta e testes com tecnologias assistivas. \\
\hline
\textbf{Facilidade de uso} &
Simples: comportamento acessível já incluso. &
Complexo: requer conhecimento técnico e implementação adicional. \\
\hline
\textbf{Necessidade de script} &
Funciona nativamente, sem necessidade de JavaScript. &
Geralmente precisa ser combinado com JavaScript para funcionar corretamente. \\
\hline
\textbf{Quando usar} &
Sempre que possível. &
Somente quando não há equivalente nativo. \\
\hline
\textbf{Exemplo} &
\texttt{<button>} (já acessível e com suporte ao teclado). &
\texttt{<div role="button">} (precisa de ARIA + JS + eventos de teclado). \\
\hline
\end{tabular}
\caption*{\textbf{Fonte}: Do Autor, 2025.}
\end{table}

Por fim, tem-se que a semântica nativa deve ser a base da acessibilidade web e o uso da WAI-ARIA deve ser tratada como uma extensão, tendo seu uso somente quando preciso, pois seu uso inadequado pode tornar a aplicação menos acessível.





\subsection{ATAG}
A ATAG (Accessibility Authoring Tools Guidelines), conforme \cite{W3C_ATAG} é uma diretriz desenvolvida pelo W3C, cujo objetivo é garantir que as ferramentas como  editores de sites, sistemas de gerenciamento de conteúdo (CMS) ou construtores de interfaces sejam acessíveis tanto para todos os perfis de usuários. Atualmente está na versão 2.0, tendo como base os princípios da versão 1.0, que são tornar as ferramentas de autoria acessíveis e ajudar os autores a criarem um conteúdo acessível.

Para tornar as ferramentas acessíveis busca-se que elas também sejam utilizadas por pessoas com deficiência, e quanto ao auxílio a criação de conteúdo acessível, o mesmo tem o foco orientativo, ou seja, facilitar a criação de produtos digitais acessíveis por padrão, através de sugestões automáticas, validações, modelos inclusivos e boas práticas embutidas nos kits de design \cite{W3C_ATAG}.

Com relação a versão 2.0 lançada em 2015, manteve-se todas as diretrizes da versão anterior, adicionando os níveis de conformidade A, AA e AAA propostos no modelo WCAG, além de uma maior integração com esse modelo, de forma a promover um conteúdo com maior acessibilidade por padrão \cite{w3c-atag20}.

O trabalho de \cite{baldiris2022evaluation} por exemplo, fez a avaliação de quatro ferramentas gratuitas e de código aberto, de autoria para criação de conteúdos educacionais, com base nas diretrizes da ATAG, avaliando a acessibilidade do conteúdo educacional gerado por essas ferramentas, com base nas recomendações da WCAG. Como conclusão estabeleceram uma série de recomendações com o objetivo de ajudar a reduzir algumas lacunas relacionadas à acessibilidade.
        .
Assim, através da utilização dos kits que seguem a ATAG garante-se que o conteúdo acessível não seja algo que deixe para ser feito posteriormente ao desenvolvimento padrão, mas sim que seja algo feito de forma integrada, evitando-se o retrabalho e também a adequação as conformidades legais, além de também ampliar o alcance das ferramentas.


% A \autoref{tab:comparative} destaca os principais elementos comparativos entre a WCAG, Marco Civil da Internet e ABNT NBR.

\begin{table}[ht]
\centering
\small % ou \footnotesize, \scriptsize, \tiny
\caption{Comparação entre WCAG, Marco Civil da Internet e ABNT NBR}
\label{tab:comparative}
\begin{tabular}{p{4cm} p{3cm} p{3cm} p{3cm}}
\toprule
\textbf{Critério} & \textbf{WCAG} & \textbf{Marco Civil da Internet} & \textbf{ABNT NBR 17225:2023} \\
\midrule
Origem & Internacional (W3C) & Nacional (Brasil) & Nacional (Brasil) \\
Tipo & Diretriz técnica internacional & Lei geral sobre uso da internet & Norma técnica (ABNT) e legislação de inclusão \\
Foco principal & Acessibilidade de conteúdo web (HTML, interfaces, apps) & Direitos digitais, privacidade e acesso universal à internet & Acessibilidade digital para pessoas com deficiência \\
Aplicação & Referência global para acessibilidade web & Garantia de acesso universal e equitativo à internet & Aplicação técnica para sites, apps e portais públicos e privados \\
Obrigatoriedade legal & Não obrigatória, mas recomendada e usada como base & Obrigatória para provedores e governo & Obrigatória quando citada por leis ou exigida em contratos públicos \\
Relacionamento entre si & Base para a ABNT NBR 17225:2023 & Prevê o acesso universal, apoiando a inclusão digital & A NBR 17225 detalha tecnicamente as exigências legais da LBI \\
Direitos abordados & Acesso a conteúdos digitais por pessoas com deficiência & Privacidade, neutralidade da rede, liberdade de expressão & Acesso igualitário, não discriminação, adaptação de plataformas digitais \\
\bottomrule
\multicolumn{4}{l}{\small \textbf{Fonte}: Do Autor, 2025} \\
\end{tabular}
\end{table}

A tabela compara três normativas centrais para a acessibilidade digital no Brasil: WCAG, Marco Civil da Internet e ABNT NBR 17225:2023. A WCAG, de origem internacional, serve como base técnica para acessibilidade web, sendo amplamente adotada, embora não obrigatória. O Marco Civil, lei nacional, garante o acesso universal à internet e fundamenta juridicamente a inclusão digital. Já a ABNT NBR 17225:2023 traduz essas diretrizes e princípios legais em requisitos técnicos aplicáveis a sites e aplicativos no contexto brasileiro. Juntas, essas normativas se complementam ao combinar diretrizes técnicas, garantias legais e orientações práticas para promover a acessibilidade digital.


\section{Leitores de tela e sua interpretação de interfaces}

De acordo com \citeonline{incaper2023}, os leitores de tela são usados majoritariamente por pessoas com maiores graus de deficiência visual, sendo muitas vezes a principal forma de acesso aos conteúdos digitais, e sua função é basicamente ler ou traduzir o conteúdo que está na interface para uma saída em áudio ou linhas braille. Alguns dos modelos mais conhecidos são o NVDA (\textit{NonVisual Desktop Access}), o JAWS (\textit{Job Access With Speech}) e o VoiceOver que está presente nos dispositivos Apple.

Dessa forma, e conforme \citeonline{matuzovic2024_accessibility_cookbook}, os leitores de tela trabalham com o conceito de árvores de acessibilidade, que é uma estrutura paralela a árvore DOM (\textit{Document Object Model}), que é gerada pelos navegadores e representa os elementos de uma página da web com as informações relevantes para tecnologias assistivas. Porém, enquanto a árvore DOM contém todos os elementos HTML, a árvore de acessibilidade filtra e transforma apenas os elementos visíveis e semanticamente relevantes, incluindo informações como labels, texto alternativo nas imagens, função dos elementos (botão, imagem, título, link), estado atual (ativo, selecionado, expandido), rótulos em campos de formulário, entre outros.

A geração da árvore de acessibilidade é baseada na semântica nativa do HTML, atributos ARIA e estilos CSS que muitas vezes afetam a visibilidade dos elementos na tela, e nisso começa o problema do uso dos leitores de tela, visto que eles fazem a análise da estrutura do código HTML e transmitem ao usuário, de forma linear, as informações contidas na página, assim, o correto funcionamento desses leitores depende de forma significativa como o código HTML foi criado e se ele possui os elementos que permitem sua leitura. Um exemplo claro é o uso da tag alt dentro das imagens, que faz a descrição visual do conteúdo da imagem, e que só é utilizado por esses leitores de tela e também caso a imagem não tenha sido carregada por algum problema de conexão ou por não existir.

Outros elementos como <header>, <nav>, <main>, <article> e <footer> informam ao leitor de tela qual a função daquela parte da página, permitindo que o usuário navegue por seções com maior fluidez e entendimento, visto que essas ferramentas funcionam de forma linear, o que também pode ser um tanto quanto demorado para ler páginas grandes ou com muito conteúdo, como publicidades em geral. Da mesma forma, utilizar <button> no lugar de <div onclick="..."> ou <label> vinculado a campos de formulário são práticas que ajudam o leitor de tela a identificar corretamente ações e entradas de dados.

O trabalho de \citeonline{incaper2023} e \citeonline{wai2017} apresenta uma série de dicas e boas práticas para a correta interpretação do código por leitores de tela, sendo as principais citadas abaixo, e além disso, a Tabela \ref{htmlxleitores} apresenta o equivalente em HTML considerando essas boas práticas de desenvolvimento para o uso de leitores de tela.

\begin{itemize}

  \item Evite links genéricos como “clique aqui”. Use descrições completas e, quando necessário, adicione o atributo \texttt{title} para fornecer informações complementares.

  \item Não faça uso de tabelas com células mescladas.

  \item Evite usar imagens para representar textos que poderiam ser escritos.

  \item Gerencie corretamente o foco com \texttt{tabindex}, permitindo navegação fluida por teclado. Use \texttt{tabindex="-1"} para elementos que devem receber foco programaticamente, mas não na ordem natural de tabulação.

  \item Em apresentações (PowerPoint, por exemplo), organize a ordem de tabulação dos elementos, utilize fontes sem serifa de tamanho mínimo 32pt e mantenha bom contraste entre texto e fundo.

  \item O texto deve ser alinhado à esquerda (sem justificação) para facilitar a leitura por pessoas com deficiência visual.

  \item Hiperlinks devem ser visualmente distintos do texto comum e apresentar indicadores visuais ao foco (por mouse ou teclado).

  \item Títulos de páginas devem ser curtos, objetivos e iniciar com as informações mais importantes, bem como utilizar corretamente os estilos de títulos (h5, h4, etc) para que leitores de tela reconheçam a hierarquia da informação.

  \item Estruture o conteúdo com títulos e subtítulos marcados com os estilos nativos do editor (ex.: Título 1, Título 2).

  \item Evite espaços em branco excessivos para simular mudança de página. Use comandos de quebra de página apropriados (como \texttt{CTRL + Enter}).

  \item Insira links clicáveis no sumário de documentos digitais para facilitar a navegação.

  \item Sempre teste seus conteúdos com leitores de tela populares, como NVDA (Windows), VoiceOver (dispositivos Apple) ou Orca (Linux), para garantir que estão acessíveis na prática.

  
\end{itemize}

Por fim, os autores \citeonline{incaper2023} ainda sugerem que não se utilize apenas de um meio visual para passar uma informação, como no caso da imagem na Figura \ref{gato_fofo}, em que se utiliza o elemento visual das cores, mas também o elemento textual dos números.

\begin{figure}[htb] 
    \centering
    \caption{Uso de mais um elemento visual para representar o conteúdo}
    \includegraphics[width=1.0\textwidth]{Figuras/cap2_img_1.PNG}
    \label{gato_fofo}
    \caption*{Fonte: \cite{incaper2023}}
  \end{figure}

  Abaixo também apresenta-se a descrição de como ocorre a leitura dos elementos HTML, de forma geral, conforme \cite{matuzovic2024_accessibility_cookbook}:

\begin{itemize}

  \item Títulos (<h1>, <h2>): Os leitores de tela informam a hierarquia dos títulos e isso permite que os usuários tenham uma visão estruturada da página e possam navegar entre seções com atalhos.\\
    Exemplo verbalizado:\\
    \textit{“Heading level 1: Bem-vindo ao site”}
  \item Imagens (<img>)\\
    Se houver atributo alt:\\
        \textit{“Image: logotipo da empresa”}\\
    Se alt estiver vazio (alt=""):\\
        \textit{A imagem é ignorada }\\
    Se alt estiver ausente:\\
        \textit{Leitores podem ler o nome do arquivo (o que não é recomendável).}\\
    \item Regiões e marcos (<nav>, <main>, <aside>): são elementos que permitem a navegação rápida por partes da página\\
    \textit{“Navigation region”, “Main content”, “Complementary content”}
     \item Formulários sem rótulo/ label (<input> sem <label> ou aria-label): são usados para que os usuários entendam o propósito do campo\\
    \textit{“Edit text” (sem contexto)}\\
    Com rótulo:\\
    “Search, edit text”
\end{itemize}

Ainda, é possível executar a leitura contínua, onde o leitor de tela percorre sequencialmente os elementos da árvore de acessibilidade, sendo similar à leitura linear de um documento, o que é ideal para leituras de textos longos. %conforme exemplo abaixo \cite{matuzovic2024_accessibility_cookbook}.
%   \textit{ “Heading level 1: Bem-vindo ao site. Paragraph: Este site oferece...”
%Outro recurso fundamental são os atributos ARIA (\textit{Accessible Rich Internet Applications}), desenvolvidos para aumentar a acessibilidade em interfaces mais complexas (\cite{wai-aria}). Nesse caso, atributos como aria-label, aria-labelledby, aria-hidden ou role="alert" comunicam ao leitor de tela o propósito ou estado de elementos que não são naturalmente compreendidos via HTML. Por exemplo, uma div estilizada como botão visualmente pode ser anunciada corretamente se receber o atributo role="button" e responder aos eventos esperados do teclado.}

Entretanto, a má utilização de ARIA ou a ausência de estrutura semântica adequada pode causar várias consequências, como ao invés de facilitar a leitura, pode gerar confusão e fazer com que elementos "invisíveis" não sejam encontrados, ou até mesmo fazer com que menus sejam lidos simplesmente como blocos de texto, sem as opções de link e navegação que um menu deveria possuir.

Devido a esses problemas de sua má utilização, um de seus princípios básicos é “não use ARIA se puder usar HTML nativo”, pois os elementos semânticos da linguagem já trazem embutidos os comportamentos e a acessibilidade esperados pela maioria dos leitores de tela \cite{wicaria2023}.

\begin{table}[H]
\centering
\small
\caption{Exemplos de boas práticas para leitores de tela}
\begin{tabular}{|p{6cm}|p{9.3cm}|}
\label{htmlxleitores}
\hline
\textbf{Código comum em HTML} & \textbf{Melhoria com acessibilidade} \\
\hline
\texttt{<div onclick="enviar()"> Enviar </div>} & 
\texttt{<button onclick="enviar()"> Enviar </button>}\\
& Uso do elemento nativo \texttt{<button>} que permite foco por teclado e melhor interpretação por leitores de tela. \\
\hline
\texttt{<img src="logo.png">} & 
\texttt{<img src="logo.png" alt="Logotipo da empresa XYZ">}\\
& O atributo \texttt{alt} fornece descrição textual da imagem, essencial para leitores de tela e para quando a imagem, por algum motivo, não puder ser carregada. \\
\hline
\texttt{<div>Menu principal</div>} & 
\texttt{<nav aria-label="Menu principal">...</nav>}\\
& Uso do elemento semântico \texttt{<nav>} com atributo \texttt{aria-label} para melhorar a navegação e semântica da página. \\
\hline
\texttt{<span>Erro!</span>} & 
\texttt{<div role="alert">Erro ao enviar o formulário.</div>}\\
& O atributo \texttt{role="alert"} faz com que o leitor de tela anuncie automaticamente mensagens importantes. \\
\hline
\texttt{<ul><li>...</li></ul>} (sem título) & 
\texttt{<h2 id="produtos">Produtos</h2>} \\
& \texttt{<ul aria-labelledby="produtos">...</ul>} \\
& O atributo \texttt{aria-labelledby} associa o título à lista, facilitando a contextualização para leitores de tela. \\
\hline
\texttt{<a href="doc.pdf">Clique aqui</a>} & 
\texttt{<a href="doc.pdf" title="Baixar regulamento em PDF">Regulamento (PDF)</a>} \\
& O atributo \texttt{title} torna o propósito do link mais claro. \\
\texttt{<div>Rodapé</div>} & 
\texttt{<footer>Rodapé</footer>} \\
& O uso de \texttt{<footer>} identifica semanticamente a seção de rodapé da página. \\
\hline
\texttt{<input type="text">} & 
\texttt{<label for="nome">Nome:</label>} \\
& \texttt{<input id="nome" type="text">} \\
& O uso de \texttt{<label>} vinculado ao \texttt{input} permite que o leitor de tela anuncie o campo corretamente. \\
\hline
\texttt{<i class="fa fa-star"></i>} (ícone decorativo) & 
\texttt{<i class="fa fa-star" aria-hidden="true"></i>} \\
& Ícones puramente decorativos devem ser ocultos dos leitores com \texttt{aria-hidden="true"}. \\
\hline
\texttt{<div tabindex="0">...</div>} sem função clara & 
\texttt{<section tabindex="-1">...</section>} \\
& O uso adequado de \texttt{tabindex} permite navegação controlada por teclado, quando necessário. \\
\hline
\end{tabular}
\caption*{\textbf{Fonte}: Do Autor, 2025.}
\end{table}

Além disso, os leitores de tela percorrem o conteúdo de maneira sequencial, o que significa que a ordem dos elementos no código influencia diretamente na experiência do usuário, o que reforça a importância de se planejar a estrutura do HTML como uma narrativa acessível, onde títulos, listas, botões e formulários seguem uma ordem lógica e consistente, tal como demonstrado na Figura \ref{tags_semantica}, onde antes mesmo de visualizar o resultado em tela, faz-se ideia de o que significa, aonde estará posicionado e o objetivo do trecho.

\begin{figure}[H] 
    \centering
    \label{tags_semantica}
    \caption{Comparação de tags semânticas em código HTML}
    \includegraphics[width=1.0\textwidth]{Figuras/cap2_img_2.PNG}
    \label{tags_semantica}
    \caption*{Fonte: \cite{abnovato2021html}}
  \end{figure}

Por fim, a pesquisa de \citeonline{nttdata}, realizada pela NTT Data em 2022 fez um levantamento dos principais leitores de tela utilizados no Brasil, obtendo 564 respostas válidas, sendo que a maior parte dos usuários de desktop utilizam o leitor de tela NVDA com Chrome ou NVDA com Firefox, e quanto aos mobile, os leitores mais utilizados são TalkBack com Chrome, VoiceOver com Safari e JieShuo (chinês) com Chrome. A pesquisa ainda fez o levantamento demográfico por região, sexo, nível de escolaridade e faixa etária, além de como o leitor de tela é de fato utilizado (navegação por região da página, links, cabeçalhos, entre outros) e cruzamento das informações, permitindo um panorama geral do uso desses leitores no Brasil.

De forma complementar, a Tabela \ref{tab:leitores_tela} faz um resumo dos principais leitores de tela, sendo NVDA, JAWS e TalkBack.

\begin{table}[ht]
\centering
\caption{Comparação entre leitores de tela populares}
\begin{adjustbox}{max width=\textwidth}
\begin{tabular}{|l|c|c|c|c|}
\hline
\textbf{Característica} & \textbf{NVDA} & \textbf{JAWS} & \textbf{TalkBack} & \textbf{VoiceOver} \\ \hline

\textbf{Plataforma} & Windows & Windows & Android & macOS / iOS \\ \hline

\textbf{Custo} & Gratuito & Pago (licença) & Gratuito & Gratuito \\ \hline

\textbf{Voz padrão} & \textit{Microsoft Speech Platform} & Várias opções, voz humana & Google TTS & Siri \\ \hline

\textbf{Personalização} & Alta & Muito alta & Moderada & Alta \\ \hline

\textbf{Suporte a idiomas} & Multilíngue & Multilíngue & Multilíngue & Multilíngue \\ \hline

\textbf{Popularidade} & Muito usado & Muito usado (profissional) & Muito usado (mobile) & Muito usado (usuários Apple) \\ \hline

\end{tabular}
\end{adjustbox}
\label{tab:leitores_tela}
\caption*{\textbf{Fonte}: Do Autor, 2025.}
\end{table}


\section{Design de interface e documentação no Figma}

Segundo \citeonline{garrett2011} o design de interfaces é um componente essencial no desenvolvimento de produtos e serviços digitais que atendam às necessidades dos usuários de forma eficiente e inclusiva. Nos últimos anos, o Figma destacou-se como líder design de interfaces e prototipação colaborativa, com destaque pela sua plataforma baseada em nuvem que permite múltiplos usuários trabalharem de forma conjunta e colaborativa em um projeto \cite{ibrahim2023effectiveness}. Essa característica revolucionou o processo de criação, facilitando a comunicação entre equipes multidisciplinares e acelerando o ciclo de desenvolvimento.

No contexto de \textit{handoff}, o Figma facilita a entrega técnica por meio da exposição de propriedades CSS, medidas, cores e exportação de ativos. Por outro lado, esse processo apresenta limitações importantes quando se trata de acessibilidade.

Assim, uma das funcionalidades utilizadas do Figma é o suporte ao uso de \textit{UI Kits}, ou seja, coleções de componentes visuais padronizados, como botões, formulários, ícones e layouts, que garantem consistência e coerência no design \cite{ryhus2024differences}. A utilização desses kits permite uma maior produtividade e a padronização visual independente da plataforma utilizada, o que contribui para que as interfaces sejam reconhecíveis e intuitivas para os usuários, e acordo com \citeonline{garrett2011}. Além disso, eles tornam o desenvolvimento mais produtivo visto que não é necessário retrabalho na manutenção e atualização dos projetos.

Apesar das vantagens mencionadas, o Figma também enfrenta desafios e possui limitações quanto a usabilidade, pois embora existam diversos \textit{plugins} que ajudam na análise do contraste de cores ou na validação da tabulação através do teclado, por exemplo, elas ainda não estão completamente integradas ao fluxo natural de uso da ferramenta, de acordo com \citeonline{oliveira2023development}. Por causa disso, algumas atualizações na documentação dos requisitos de acessibilidade muitas vezes precisam ser feitas manualmente, o que pode contribuir para que fiquem desatualizadas ou inconsistentes.

Além disso, \citeonline{lindholm2023accessibility} informa sobre a ausência de suporte nativo a práticas inclusivas. A ferramenta também não realiza validações automáticas de acessibilidade, como verificação de contraste ou estrutura de navegação por teclado, o que exige atenção redobrada dos designers. Além disso, é comum a ausência de rótulos textuais adequados em botões e elementos interativos, o que compromete o uso por leitores de tela, especialmente em fluxos onde não há documentação paralela \cite{lindholm2023accessibility}.

Apesar disso, os pesquisadores \citeonline{kokate2022exploring} realizaram a avaliação de 10 ferramentas digitais de prototipação, estando o Figma incluso, de forma a entender os recursos de acessibilidade que disponibilizavam e descobriu-se que a acessibilidade dava-se basicamente através de \textit{plugins} terceiros, demonstrando o potencial de melhora nas mesmas.

Nesse cenário, os UI Kits desempenham um papel importante, pois através dos componentes já padronizados mantêm-se a consistência visual e os erros durante a prototipação são reduzidos. Mais importante ainda, kits bem projetados podem incorporar práticas acessíveis desde sua concepção, promovendo decisões mais responsáveis e inclusivas \cite{lindholm2023accessibility}. No entanto, a eficácia desses kits depende diretamente de sua correta adoção e da clareza de suas documentações.

Já a ausência de documentação de acessibilidade no \textit{handoff} pode gerar falhas críticas, onde os desenvolvedores podem frequentemente se deparar com componentes sem rótulos acessíveis, hierarquias de foco desorganizadas e estruturas não semânticas, o que resulta em interfaces mal adaptadas para usuários com deficiências \cite{lindholm2023accessibility}. Com base nisso, a inserção de anotações e guias visuais dentro do próprio arquivo Figma, ou o uso de kits com diretrizes explícitas, torna-se fundamental para garantir a continuidade da intenção de design ao longo do ciclo de desenvolvimento.


\section{Uso de kits e anotações visuais para acessibilidade}
Segundo \citeonline{cooper2014}, a acessibilidade digital apresentou avanço com as diretrizes propostas pelo WCAG e W3C, porém, ainda assim, sua completa e eficaz aplicação apresenta desafios e barreiras, principalmente na etapa da prototipação e do design das interfaces. Nesse cenário, os kits e anotações visuais aparecem como soluções eficazes para integrar práticas acessíveis ao fluxo de trabalho cotidiano de design, reforçando também a acessibilidade como uma etapa do projeto.



Os kits de anotação de acessibilidade (\textit{A11y Annotation Kits}), amplamente difundidos na comunidade Figma, representam uma ferramenta crucial para integrar especificações de acessibilidade diretamente nos artefatos de design, como \textit{wireframes} e protótipos. Esses kits funcionam como uma camada de documentação visual, onde anotações indicam atributos essenciais para a experiência de usuários de tecnologias assistivas, tais como a ordem de foco do teclado, a hierarquia de leitura, o contraste de cores e o uso de atributos ARIA (\textit{Accessible Rich Internet Applications}). Ao tornar essas especificações explícitas na fase de design, os kits não apenas facilitam a comunicação entre designers e desenvolvedores, mas também servem como um recurso de aprendizado para equipes multidisciplinares. Essa necessidade é corroborada pela pesquisa da \citeonline{handtalk2023}, que aponta a falta de experiência prévia em acessibilidade como um desafio recorrente entre os profissionais da área.

A descoberta de tais recursos na comunidade Figma é facilitada por buscas com termos como "A11y", "Accessibility" ou "Accessibility kit". Entre os mais proeminentes, destacam-se o \textit{A11y Annotation Kit} \citeonline{a11ykit}, focado na clareza da comunicação técnica; o \textit{Web Accessibility Annotation Kit} \citeonline{cvs_health_kit}, orientado à estrutura macro da página; o \textit{Intopia's Accessibility Annotation Kit} \citeonline{intopia_kit}, que se aprofunda na conformidade com as diretrizes WCAG; e o \textit{Pencil A11Y Kit} \citeonline{pencil_a11y_kit}, que adota uma abordagem visual distinta para fases iniciais de ideação. Esses kits, embora com abordagens diferentes, compartilham o objetivo de traduzir os requisitos abstratos de acessibilidade em especificações concretas, como demonstrado na \autoref{tab:kit_comparison_long}  e nas \autoref{cvs_health_kit} e \autoref{brainly_kit}, que ilustram a utilidade de alguns desses kits.

\begin{longtable}{@{}p{2.8cm} p{4cm} p{3.5cm} p{3.5cm}@{}}

% LEGENDA E CABEÇALHOS
\caption{Análise comparativa entre os principais kits de anotação de acessibilidade}
\label{tab:kit_comparison_long} \\
\toprule
\textbf{Kit Figma} & \textbf{Escopo das Anotações} & \textbf{Público-Alvo Principal} & \textbf{Ponto Forte Estratégico} \\ 
\midrule
\endfirsthead % Fim do cabeçalho da primeira página

\caption[]{(Continuação)} \\
\toprule
\textbf{Kit Figma} & \textbf{Escopo das Anotações} & \textbf{Público-Alvo Principal} & \textbf{Ponto Forte Estratégico} \\ 
\midrule
\endhead % Fim do cabeçalho das páginas de continuação

% RODAPÉ DE CONTINUAÇÃO (OPCIONAL)
\midrule
\multicolumn{4}{r}{\textit{Continua na próxima página...}} \\
\endfoot

\bottomrule
\addlinespace
\multicolumn{4}{c}{\small\textbf{Fonte}: Do Autor, 2025.} \\
\endlastfoot

% CORPO DA TABELA
A11y Annotation Kit & Estrutura semântica de componentes (cabeçalhos, botões, links) e fluxo de interação (ordem de foco). & Designers e Desenvolvedores Front-end. & \textbf{Simplicidade e Clareza:} Foco em traduzir visualmente a estrutura HTML e os atributos ARIA mais comuns, agilizando o \textit{handoff}. \\
\addlinespace 
Web Accessibility Annotation Kit & Estrutura macro da página, incluindo \textit{landmarks} (cabeçalho, navegação, conteúdo principal), e hierarquia de leitura. & Designers, Arquitetos de Informação e Desenvolvedores. & \textbf{Visão Estrutural:} Garante que a navegação e a compreensão global da página sejam lógicas para usuários de leitores de tela. \\
\addlinespace
Intopia's Accessibility Annotation Kit & Anotações detalhadas em nível de componente, com referências explícitas aos critérios de sucesso das WCAG. & Times de Produto, QAs e Desenvolvedores que necessitam de alta conformidade técnica. & \textbf{Rigor Técnico e Educacional:} Atua como um guia de conformidade, conectando o design às regras das WCAG e educando a equipe. \\
\addlinespace
Pencil A11Y Kit & Anotações com estilo visual "desenhado à mão" (\textit{sketch}), ideal para fases de ideação e \textit{wireframing} de baixa fidelidade. & Designers de UX/UI em fases iniciais de projeto. & \textbf{Comunicação Rápida e Informal:} Reduz a formalidade das anotações, incentivando a inclusão de considerações de acessibilidade desde o brainstorming. \\

\end{longtable}

\begin{figure}[h] 
    \centering
    \label{cvs_health_kit}
    \caption{Exemplo de aplicação do kit da CVS Health}
    \includegraphics[width=1.0\textwidth]{Figuras/A11y_cvs_health.png}
    \label{cvs_health_kit}
    \caption*{Fonte: \citeonline{cvs_health_kit}}
\end{figure}

\begin{figure}[H] 
    \centering
    \label{brainly_kit}
    \caption{Exemplo de aplicação do Pencil A11Y Kit}
    \includegraphics[width=1.0\textwidth]{Figuras/A11y_brainly.png}
    \label{brainly_kit}
    \caption*{Fonte: \citeonline{pencil_a11y_kit}}
\end{figure}

De forma complementar, a Microsoft também tem investido no design inclusivo, através de seus \textit{toolkits}, que são kits que orientam o designer na construção de soluções que considerem diferentes capacidades físicas, cognitivas e sensoriais  \cite{microsoftinclusive, inclusivebydesign2016}.

Nesse teor o trabalho de \citeonline{lindholm2023accessibility} explora como um kit de anotações para acessibilidade pode apoiar designers e comunicar especificações de acessibilidade para desenvolvedores, indo além de apenas cumprir as diretrizes e legislações vigentes. Assim, utilizando-se métodos de pesquisa qualitativa e um processo de design centrado no usuário, por meio de atividades de design como entrevistas, ideação, prototipagem e testes de usabilidade, explora-se uma ferramenta de anotação de forma a avaliar como ela poderia contribuir para o suporte e a comunicação. Como conclusão, o projeto indica que os kits de anotações têm potencial para melhorar a comunicação e apoiar os designers em seu processo de criação.

Um outro fator de inclusão considerado pelos pesquisadores \citeonline{fraga-viera2020inclusive} é a criação e modelagens de produtos que sejam verdadeira inclusivos, considerando questões como gênero, idade, valores éticos ou deficiências, propondo assim o desenvolvimento de um kit para tal.

Tem-se então que os benefícios desses kits não se restringem à etapa de design, mas se estendem até mesmo na fase de implementação, transformando-se em requisitos de acessibilidade que precisam ser de fato desenvolvidos e testados. Com base nisso, os kits atuam como facilitadores no processo de \textit{handoff} entre design e desenvolvimento \cite{bennett2019}. 

No presente contexto, os \textit{handoff} podem ser entendidos como momentos de transição de responsabilidade ou de entrega de materiais entre diferentes equipes ou profissionais, como por exemplo da fase de design para a de desenvolvimento, ou do desenvolvimento para a etapa de testes, conforme \citeonline{cooper2014}.

Por outro lado, \citeonline{bennett2019} menciona que os \textit{handoffs} tem pouca documentação clara, sendo então propensos a erros e mal entendidos por parte da equipe, principalmente caso ela seja muito grande, podendo gerar também entregas incompletas ou ambíguas e uso de ferramentas incompatíveis entre os times.

O estudo de \citeonline{bennett2019} então aponta que as falhas de comunicação entre as equipes são uma das principais causas de não conformidade com critérios de acessibilidade, mesmo quando estes foram considerados no design inicial, e uma forma de melhorar os \textit{handoffs} é justamente o uso de ferramentas como o Figma, Jira, Notion, Zeplin, entre outros, além de constantes reuniões de alinhamento, checklists para entrega e documentação enxuta e clara sobre os entregáveis esperados.

\begin{comment}
Adicionalmente, \textit{design systems} modernos têm integrado práticas acessíveis de forma sistematizada, como por exemplo "O Material Design", que inclui diretrizes específicas para cores, tipografia e foco do teclado, fornecendo componentes já com acessibilidade embutida \cite{materialdesign2025}. 

Outros kits são o \textit{Carbon Design System} da IBM e o \textit{Fluent UI}, também da Microsoft e que seguem o mesmo princípio básico, promovendo a reutilização de componentes acessíveis e escaláveis. Esses sistemas mostram como padronizações visuais podem ser vetores não apenas de consistência estética, mas de inclusão técnica e social \cite{carbondesign, fluentui}.

A Tabela \ref{tab:kits_acessibilidade_comparacao} apresenta uma breve comparação entre os kits mencionados, trazendo o diferencial de seu foco, suporte a acessibilidade, facilidade de \textit{handoff}, padronização e recursos de escalabilidade.

\begin{table}[H]
\centering
\small 
\caption{Comparação dos kits e sistemas de design para acessibilidade}
\begin{tabular}{|l|c|c|c|c|c|}
\hline
\textbf{Critérios} & \textbf{A11y Kit} & \textbf{Microsoft} & \textbf{Material} & \textbf{Carbon} & \textbf{Fluent UI} \\
\hline
Foco & Anotações & Inclusivo & UI & UI & UI \\
\hline
Suporte Acessibilidade & Alto & Alto & Alto & Alto & Alto \\
\hline
Facilidade no Handoff & Alto & Médio & Alto & Médio & Médio \\
\hline
Padronização & Médio & Alto & Alto & Alto & Alto \\
\hline
Reuso/Escalabilidade & Médio & Alto & Alto & Alto & Alto \\
\hline
\end{tabular}
\label{tab:kits_acessibilidade_comparacao}
\caption*{\textbf{Fonte}: Do Autor, 2025.}
\end{table}
\end{comment}

Nota-se que, conforme mencionado por \cite{articleBotelho}, os kits auxiliam em diversos aspectos da acessibilidade, porém ainda possuem limitações quando aos aspectos subjetivos ou contextuais da acessibilidade, sendo necessário um esforço consciente e sistêmico para assegurar que o potencial das tecnologias digitais para a inclusão seja realizado, como por exemplo, \textit{smartphones} podem ser incompatíveis com aparelhos auditivos necessários para pessoas surdas, telas sensíveis demais para quem tem deficiências motoras, e páginas web frequentemente carecem dos rótulos de texto necessários para softwares leitores de tela usados por pessoas cegas, e mesmo que cada um desses exemplos seja corrigido, a acessibilidade pode ser de curta duração se o processo de produção por trás desse hardware ou software não for ajustado, visto que o mundo digital evoluiu de forma ágil e constante. Assim, considerando o uso dos kits no início do projeto, busca-se atuar de madeira preventiva, reduzindo as correções e melhorando a qualidade final o produto, bem como reduzindo custos com retrabalho.

Além disso, \cite{rosenfield2020} menciona que esses kits também  permite o envolvimento de \textit{stakeholders} não técnicos no processo de validação e discussão a respeito da acessibilidade. Assim, a consideração de requisitos que afetam a interface diretamente torna-se mais fácil a compreensão do impacto do design sobre elas, o que é especialmente relevante em equipes grandes e multidisciplinares, em que mais áreas contribuem no projeto, como negócio, marketing ou o setor jurídico.

Outro aspecto importante é a ausência de uma cultura organizacional voltada à inclusão e acessibilidade, conforme pesquisa realizada por \cite{parthasarathy2023}, que após estudo realizado com desenvolvedores web na Índia, descobriu-se que cerca de 70\% deles nunca tinham recebido nenhum tipo de treinamento formal sobre acessibilidade, o que confirma que tal tópico muitas vezes é visto como secundário, e por falta de tempo na construção das ferramentas, é muitas vezes ignorado.

Designers e desenvolvedores de sistemas frequentemente enfrentam desafios para implementar requisitos de acessibilidade de forma clara e eficaz, especialmente quando há pouca experiência ou ferramentas específicas para auxiliar nesse processo \citeonline{handtalk2023}. 


% Com base nisso, a acessibilidade online não pode ser considerada como uma etapa adicional ou diferencial no uso dos serviços digitais, mas como um requisito essencial que deve ser pensado e planejado desde a concepção do produto, de forma a garantir o direito de acesso à todos. Algumas ferramentas de validação automática, como o WAVE e o Access Monitor, podem auxiliar no diagnóstico de problemas, além de que treinamentos voltados aos desenvolvedores e uma possível legislação que responsabilize os mesmos, pode agilizar o uso de tecnologias assistivas.

Finalmente, soluções como o Wilia destacam-se ao oferecer funcionalidades específicas aliadas a padronização visual, documentação acessível e facilidade de exportação de dados para plataformas de desenvolvimento. Com isso, essas ferramentas representam a convergência entre o design acessível e a engenharia prática, o que visa promover uma cultura de acessibilidade como prática contínua e não apenas como etapa final ou obrigatoriedade legal.



%trabalhos relacionados sendo adicionados ao longo do texto.


\chapter{Metodologia} 
\label{cap3_metodologia} 



O presente capítulo descreve o percurso metodológico empregue na conceção, desenvolvimento e estruturação do Wilia, um kit de anotação de acessibilidade para a ferramenta Figma. A pesquisa classifica-se como Aplicada, por se focar na criação de um artefacto destinado à solução de um problema prático no fluxo de trabalho de produtos digitais. A abordagem utilizada foi Qualitativa, centrada na análise de diretrizes, na construção de um kit de componentes de documentação e na produção de conteúdo técnico de apoio. O processo foi segmentado em quatro fases sequenciais.



\section{Fundamentação Teórica e Delimitação do Escopo de Atuação}

O ponto de partida da investigação foi a identificação de uma lacuna comunicacional e ferramental entre as disciplinas de design e desenvolvimento no que tange à especificação de requisitos de acessibilidade. Para contextualizar a relevância deste problema no cenário brasileiro, a etapa inicial incluiu a análise de pesquisas e relatórios de organizações nacionais proeminentes na área. Foram consultados os levantamentos periódicos realizados pelo \citeonline{mwpt_2024} (Movimento Web para Todos) em parceria com a BigData Corp, bem como a pesquisa “Panorama da Acessibilidade Digital no Brasil”, publicada pela \citeonline{handtalk2023}. Ambos os estudos serviram para evidenciar a dimensão e a persistência das barreiras digitais no ecossistema web nacional, reforçando a relevância e a urgência do problema abordado por este trabalho.

Esta análise de cenário nacional, somada ao estudo das fontes normativas globais, compôs o embasamento teórico do projeto. A investigação aprofundada foi direcionada por três pilares:
\begin{enumerate}
\item 
As Diretrizes de Acessibilidade para Conteúdo Web (WCAG), do World Wide Web Consortium (W3C), que serviram como a base normativa para todos os critérios técnicos.
\item 
Os relatórios anuais da WebAIM (Web Accessibility in Mind), cuja análise dos erros mais comuns na web permitiu um diagnóstico preciso dos pontos de maior fragilidade em interfaces digitais.
\item 
A documentação técnica do MDN Web Docs (Mozilla) e as práticas de autoria do WAI-ARIA (W3C), que forneceram os subsídios para os exemplos de implementação e a tradução das diretrizes em código prático.
\end{enumerate}

A síntese deste levantamento resultou na delimitação do escopo do artefato. Foram selecionados para compor o kit os componentes de anotação que correspondem às áreas de maior incidência de falhas de acessibilidade, a saber: Regiões (Landmarks), Cabeçalhos, Botões, Campos de Entrada, Imagens, Hiperlinks, Ordem de Leitura, Ordem de Foco, Anotação Geral e a instrução para Ignorar elementos decorativos.



\section{Análise Comparativa de Artefatos e Modelagem Estratégica}

Uma vez definido o escopo, a segunda fase concentrou-se na análise de soluções correlatas por meio de um processo de benchmarking conduzido diretamente na plataforma Figma Community, sendo a biblioteca de conteúdos da comunidade. A pesquisa foi realizada utilizando termos como “acessibility”, “a11y”, “Acessibilidade” e “Acessibility kit”, o que levou à seleção de artefatos de alto impacto e ampla utilização pela comunidade de design, estabelecendo-os como referências para análise. 

A seleção teve como critérios o quão bem ranqueados os artefatos estavam nos resultados da busca, o número de duplicações, que remete ao número de usuários que utilizam ou utilizaram o artefato, o número de marcações como “gostei” que é uma métrica de dizer se quem o usou gostou ou não e, principalmente, se o artefato abrangia o material de estudo dessa pesquisa que trata sobre leitores de tela e sistemas e sites web. Visto isso, foram selecionados os artefatos previamente citados:

\begin{itemize}
  \item \textbf{A11y Annotation Kit}: Com uma Utilização pela comunidade de mais de 14900 suários e marcado como gostei por mais de 1100 usuários.
  \item \textbf{Web Accessibility Annotation Kit}: Com uma Utilização pela comunidade de mais de 9800 suários e marcado como gostei por mais de 1000 usuários.
  \item \textbf{Intopia's Accessibility Annotation Kit}: Com uma Utilização pela comunidade de mais de 2000 suários e marcado como gostei por mais de 135 usuários.
  \item \textbf{Pencil A11Y Kit}: Com uma Utilização pela comunidade de mais de 1200 suários e marcado como gostei por mais de 80 usuários.
\end{itemize}


\section{Concepção do Kit no Figma}

Esse último ponto consistiu na construção efetiva do projeto Wilia no Figma, concebido como uma solução de design final para documentação. A construção seguiu uma sequência rigorosa de etapas processuais: primeiramente, a definição da arquitetura visual e nomenclatura do kit; em seguida, a construção da base de componentes fundamentais; posteriormente, a estruturação destes elementos em anotações complexas; na sequência, o desenvolvimento técnico do “Multicomponente” centralizador; e, por fim, a curadoria e integração do conteúdo educacional na ferramenta.

O primeiro passo foi projetar a identidade visual do próprio kit de anotação, definindo a Arquitetura Visual e Nomenclatura. Foi desenvolvida uma paleta de cores, tipografia e iconografia distintas, para garantir que as anotações do Wilia fossem sempre legíveis e visualmente apartadas do design que estivessem documentando. Simultaneamente, foi estabelecida uma convenção de nomenclatura para todos os componentes e suas camadas, assegurando a escalabilidade e manutenibilidade do arquivo.

A construção da base de componentes foi guiada por um princípio de componentização modular. Esta etapa iniciou-se pela criação dos elementos mais fundamentais e indivisíveis. Estes elementos-base, como a Etiqueta (o rótulo visual), o contêiner de Nota (a área de texto) e os Conectores (as linhas de ligação), foram criados como componentes mestres no Figma.

Na etapa de estruturação das anotações, esses elementos-base foram então combinados para formar estruturas mais complexas e funcionais. O componente Nota, por exemplo, foi estruturado com a ferramenta Auto Layout do Figma para se adaptar dinamicamente ao conteúdo. Foram criados campos de texto específicos para “Título da anotação”, “Descrição da nota”, e “Código de exemplo”, conforme a anatomia do kit. Esta estruturação garantiu que cada anotação fosse um conjunto coeso e responsivo.

O desenvolvimento do “Multicomponente” com Component Properties foi a solução técnica implementada para otimizar a usabilidade do kit, conforme planejado na estratégia. Sua implementação técnica no Figma envolveu a criação de um componente base contendo todas as possíveis anotações; a configuração de uma propriedade de variante chamada “Tipo de Anotação”, com opções para cada um dos dez tipos de especificação (Cabeçalho, Botão, Imagem, etc.); e o uso de propriedades de texto e booleanas (Component Properties) para permitir ao usuário final a edição de textos e a alternância da visibilidade de seções diretamente no painel de inspeção do Figma.

Finalmente, a etapa de curadoria e integração do conteúdo educacional foi o que efetivamente transformou a coleção de elementos visuais em um sistema de conhecimento integrado. Com a estrutura técnica finalizada, foi realizado o processo de pesquisa, redação e integração do conteúdo de apoio. Para cada variante dentro do “Multicomponente”, o texto correspondente à sua documentação (Definição, Impacto no Usuário, Diret





\begin{comment}
\section{ Arquitetura da Informação e Construção do kit}

A fase final do projeto foi dedicada à produção do conteúdo educacional que acompanha cada componente, transformando o artefato de uma ferramenta para um recurso de consulta. Para cada um dos tipos de anotação, foi executado um procedimento sistemático de curadoria e redação, que resultou em um guia detalhado, padronizado com a seguinte estrutura:
\begin{description}
  \item[Definição e Impacto no Usuário:] 
  Uma explanação do conceito e sua importância, frequentemente ilustrada por cenários de uso contrastantes ("Cenário Positivo" e "Cenário Negativo") para evidenciar o impacto na experiência de usuários de tecnologias assistivas.
  \item[Diretrizes para Design e WCAG:]
  Orientações práticas para a aplicação da anotação, sempre vinculadas aos critérios normativos correspondentes das Diretrizes de Acessibilidade para Conteúdo Web.
  \item[Exemplos de Implementação:] 
  Foram incluídos fragmentos de código (HTML e ARIA) e diretrizes técnicas para os desenvolvedores, sintetizados a partir das melhores práticas preconizadas pela W3C e MDN.
\end{description}

Todo este material textual foi meticulosamente inserido na documentação interna de cada componente no Figma, acessível através das páginas de cada componente, assegurando que a fundamentação teórica estivesse sempre a um clique de distância da prática de design. Para os profissionais de implementação, os exemplos e materiais de apoio para implementação foram inseridos diretamente na descrição de cada componente.
\end{comment}

\begin{comment}
{\color{red}
Seção Antiga

A primeira fase do projeto se deu pela definição de escopo, em que o ponto de pesquisa inicial foi o estudo aprofundado das Diretrizes de Acessibilidade para Conteúdo Web (WCAG 2.2), com foco no nível AA, além dos princípios da ARIA (\textit{Accessible Rich Internet Applications}). Esse embasamento técnico serviu como guia para compreender os critérios essenciais de acessibilidade que deveriam ser atendidos no kit. Além disso, analisou-se os principais erros recorrentes em acessibilidade apontados pela WebAIM, com ênfase em contextos semânticos, estrutura de sites e más práticas recorrentes. Essa análise permitiu identificar os componentes mais problemáticos em interfaces digitais, como: botões, cabeçalhos, imagens e campos de entrada, e com isso definir o escopo inicial do kit.

Também foram consultados kits de acessibilidade já existentes, observando boas práticas de documentação e apresentação, e esse processo auxiliou na modelagem de uma proposta mais clara, educativa e funcional, tanto para designers quanto para desenvolvedores.

Em seguida, a fase de prototipação foi conduzida no Figma, com uma abordagem centrada no Atomic Design, onde os componentes foram divididos em átomos (elementos simples como botões), moléculas e organismos (combinações mais complexas de elementos), permitindo uma estrutura escalável e de fácil entendimento. O grande diferencial deste projeto está na metodologia voltada não apenas à acessibilidade em si, mas também no suporte direto à atuação do designer e do desenvolvedor, visto que o kit foi pensado para ser versátil dentro do próprio Figma, oferecendo componentes que geram múltiplas variantes com facilidade. Também foram criados os chamados multicomponentes, que nada mais são que elementos que funcionam como \textit{containers} de todos os componentes relacionados, permitindo que o usuário copie apenas um deles e faça sua adaptação, modificação ou troca de suas partes diretamente, conforme a necessidade, sendo que cada variante possui sua própria documentação de apoio técnico para desenvolvedores.

O próximo passo foi a organização do kit, em que o mesmo foi dividido em diferentes seções dentro do Figma, sendo:
   \begin{itemize}
    \item \textbf{Página de Introdução:} Apresenta a proposta da \textit{Wilia} (nome dado ao kit), explicando seu objetivo e a importância da acessibilidade no design digital;

    \item \textbf{Página de \textit{Overview}:} Reúne todos os componentes em um só lugar para facilitar a visualização geral do kit;

    \item \textbf{Páginas Individuais por Componente:} Cada componente possui uma página própria onde estão organizadas as "etiquetas" de acessibilidade, ou seja, anotações visuais que indicam como aplicar as boas práticas de forma correta.

    \item \textbf{Especificações Visuais:} Também foram incluídas páginas com as estruturas visuais do kit, como paleta de cores, tipografia e ícones.
\end{itemize}

Finalmente, cada componente e variante conta com uma documentação específica, seguindo um modelo padronizado que inclui a descrição do comportamento acessível esperado, uma explicação sobre a leitura feita por leitores de tela e outras tecnologias assistivas, orientações práticas para designers e desenvolvedores, exemplos de marcação HTML semântica, e quando necessário, alternativas com ARIA, identificação da diretriz específica da WCAG a que o componente atende, e por fim, um guia explicativo sobre como aplicar as etiquetas corretamente e obter o máximo valor do kit. Além disso, para facilitar o uso técnico, o Figma foi estruturado de forma que, ao selecionar um componente, o desenvolvedor tenha acesso imediato ao texto de apoio na aba lateral, explicando suas funcionalidades e instruções de uso.
}
\end{comment}

\chapter{Resultados} 
\label{cap4_resultados} 

O principal resultado deste trabalho é o \textit{Wilia} (acrônimo para Web Accessibility Annotation Kit), um sistema de especificação visual desenvolvido na plataforma Figma. O \textit{Wilia} está disponível publicamente em \url{https://www.figma.com/design/2hN2ajzPoqDtoGcsAl0gxN/Wilia--Web-Accessibility-Annotation-Kit} para consulta.

O nome \textbf{Wilia} foi cuidadosamente selecionado para refletir a missão central do artefato. Trata-se de um acrônimo para \textbf{W}eb Accessib\textbf{ili}ty \textbf{A}nnotation Kit, destacando em sua própria nomenclatura a tríade de elementos que o compõem: a Web como ambiente de aplicação, a Acessibilidade (\textit{Accessibility}) como valor fundamental e o Kit de Anotação como formato. Essa escolha nominal visa não apenas fornecer uma identidade única e memorável, mas também encapsular o propósito de que o kit atua como um sistema dedicado a traduzir diretrizes complexas de inclusão digital em ferramentas de design de uso prático e diário.

A concepção do kit está rigorosamente baseada nos padrões técnicos da WCAG 2.2 e WAI-ARIA, e enriquecida por boas práticas de mercado.

Sobre os artefatos selecionados como comparativo e modelagem que basearam o \textit{Wilia}, foi realizada uma análise comparativa a respeito da estrutura, utilização e disponibilidade dos recursos. A análise revelou que, embora fossem funcionalmente competentes, apresentavam lacunas estratégicas:
\begin{description}
    \item [Documentação Dissociada:] A fundamentação teórica (o "porquê" de uma anotação) não estava muito bem difundida nos kits, exigindo que o usuário buscasse conhecimento em fontes externas e interrompesse seu fluxo de trabalho.

    \item [Foco na Ferramenta, Não no Conhecimento:] Os kits forneciam o meio para rotular um elemento, mas não educavam o designer sobre o impacto daquela decisão na experiência de um usuário de tecnologia assistiva.

    \item [Fluxo de Trabalho Fragmentado:] A aplicação das anotações frequentemente exigia que o usuário procurasse e gerenciasse os componentes individuais dispersos na biblioteca, tornando o processo lento e suscetível a inconsistências.
\end{description}

Essas lacunas contribuíram para a modelagem estratégica do Wilia, possibilitando, como seu grande diferencial, a funcionalidade de atuar como uma ponte entre a teoria e a prática. O \textit{Wilia} traduz normas de acessibilidade em componentes visuais e diretrizes acionáveis, criando uma ferramenta de aplicação direta e intuitiva que pode se integrar ao fluxo de trabalho de designers e desenvolvedores, capacitando-os a criar interfaces mais inclusivas desde a sua concepção.

A construção do \textit{Wilia} passou por uma série de etapas interdependentes, sendo o primeiro resultado a elaboração de uma página introdutória no projeto no Figma. Dessa forma, na seção inicial apresentam-se os objetivos do kit, o público-alvo, e o contexto de sua criação, bem como links para as documentações e diretrizes mencionados acima. A introdução foi planejada como um diálogo com o usuário que irá usar o kit, buscando com isso repassar também o valor social da acessibilidade e o que motivou a criação do kit.

A página de \textit{Visão geral}, conforme trecho recortado e demonstrado na \autoref{wilia_1}, que oferece uma visão global de todos os componentes criados, permitindo otimizar o fluxo de trabalho ao possibilitar que o usuário visualize, selecione e compreenda rapidamente a função de cada item disponível. Essa centralização facilita tanto a navegação quanto o entendimento do conjunto como um sistema coeso. Além disso, a \autoref{wilia_2} demonstra a anatomia de cada componente, projetados em diferentes formatos, tamanhos e direções justamente para deixar a ferramenta flexível e com isso, permitir um maior uso por parte dos desenvolvedores e designers. 

\begin{figure}[H]
    \centering
    \caption{Trecho da visão geral do Wilia.} % Nome da figura em cima
    \includegraphics[width=1.0\textwidth]{Figuras/wilia_1.png}
    \caption*{\small Fonte: do Autor, 2025} % Fonte abaixo da imagem
    \label{wilia_1}
\end{figure}

\begin{figure}[H]
    \centering
    \caption{Anatomia dos componentes Wilia.}
    \includegraphics[width=1.0\textwidth]{Figuras/wilia_2.png}
    \caption*{\small Fonte: do Autor, 2025}
    \label{wilia_2}
\end{figure}

  
Partindo da visão geral, o kit é estruturado em páginas dedicadas a cada categoria de componente (regiões, cabeçalhos, botões, campos de entrada, imagens, hiperlinks, ordem de leitura e ordem de foco). Cada página contém suas respectivas etiquetas de acessibilidade, que funcionam como anotações visuais aplicadas diretamente sobre os protótipos.

O propósito dessas etiquetas é destacar elementos que requerem documentação de acessibilidade específica, como o uso semântico ideal, a estrutura para leitores de tela e as interações esperadas via teclado. Essa abordagem estratégica proporciona um aprendizado contextualizado e incentiva a aplicação de boas práticas desde a fase de concepção visual. A apresentação visual destas páginas, com suas diretrizes, encontra-se no (\autoref{anexoA}) deste trabalho.

Um resultado significativo deste projeto é a concepção do componente Multiespecificação. Trata-se de um componente mestre, construído com variantes no Figma, que encapsula todos os tipos de etiquetas disponíveis no Wilia Kit. Na \autoref{willia-multicomponente} é possível ver sua arquitetura que permite que o usuário, através do painel de propriedades, alterne dinamicamente uma única instância para a especificação desejada. Isso possibilita que um ativo sirva a múltiplos contextos de documentação, representando um avanço em eficiência e usabilidade para o sistema de anotação.

\begin{figure}[H] 
    \centering
    \caption{Exemplo de alteração de uma etiqueta Título para uma Região com um componente de multiespecificação} % título em cima
    \includegraphics[width=0.8\textwidth]{Figuras/willia-multicomponente.png}
    \caption*{\small Fonte: do Autor, 2025} % fonte abaixo da imagem, sem numeração
    \label{willia-multicomponente}
\end{figure}

Cada componente foi acompanhado de uma documentação técnica incorporada ao próprio Figma, aparecendo em sua aba lateral contendo orientações práticas, exemplos de marcação HTML, explicações de comportamento esperados em tecnologias assistivas e referências às diretrizes específicas da WCAG que são atendidas. Essa abordagem pode ser vista na \autoref{willia-exemplo-tecnico} onde trouxemos um entendimento prévio sobre a especificação que pode agilizar o processo de desenvolvimento com acessibilidade.

\begin{figure}[H] 
    \centering
    \caption{Exemplo de uma incorporação técnica vista na aba de propriedades de uma etiqueta}
    \includegraphics[width=0.8\textwidth]{Figuras/Exemplo-implementação.png}
    \caption*{\small Fonte: do Autor, 2025}
    \label{willia-exemplo-tecnico}
\end{figure}

Todas as páginas possuem uma descrição textual indicando o cenário onde o Wilia é utilizado, e o cenário onde o mesmo não é usado, para indicar aos designers e desenvolvedores o ganho em acessibilidade com o uso do mesmo. Demonstrativamente a \autoref{wilia_11} possui os cenários mencionados referentes ao item Região.

\begin{figure}[H] 
    \centering
    \caption{Exemplo de descrição de cenários da etiqueta Região} % título em cima
    \includegraphics[width=0.8\textwidth]{Figuras/wilia_11.png}
    \caption*{\small Fonte: do Autor, 2025} % fonte abaixo da imagem, sem numeração
    \label{wilia_11}
\end{figure}

Com essa estrutura, o Wilia consolida-se como um recurso funcional e dinâmico para equipes multidisciplinares, promovendo uma integração mais fluida entre design e desenvolvimento, além de contribuir para a disseminação de uma cultura de acessibilidade aplicada, prática e tecnicamente fundamentada.

Por fim, para ilustrar sua aplicação prática, a seguir apresenta-se dois exemplos de uso do Wilia na documentação de uma tela acessível, sendo essas a \autoref{documentacao-exemplo-wilia} e \autoref{ordem-navegacao=foco}. A ferramenta permite descrever visualmente e tecnicamente os padrões de acessibilidade necessários, tornando o processo de implementação mais claro e replicável por toda a equipe.

\begin{figure}[H] 
    \centering
    \caption{Exemplo de documentação de acessibilidade com o Wilia para semântica} % título em cima
    \includegraphics[width=0.8\textwidth]{Figuras/documentacao-exemplo-wilia.png}
    \caption*{\small Fonte: do Autor, 2025} % fonte abaixo da imagem, sem numeração
    \label{documentacao-exemplo-wilia}
\end{figure}

\begin{figure}[H] 
    \centering
    \caption{Exemplo de documentação de acessibilidade com o Wilia para ordem de leitura e de foco} % título em cima
    \includegraphics[width=0.5\textwidth]{Figuras/ordem-navegacao=foco.png}
    \caption*{\small Fonte: do Autor, 2025} % fonte abaixo da imagem, sem numeração
    \label{ordem-navegacao=foco}
\end{figure}
\chapter{Conclusão} 
\label{cap5_conclusao} 

A implementação da acessibilidade digital é, simultaneamente, uma obrigação ética, legal e técnica, e um dos maiores desafios para as áreas de design e desenvolvimento web no cenário global. A prática corrente demonstra uma desconexão preocupante entre a existência de diretrizes internacionais e sua efetiva implementação. Análises quantitativas de larga escala confirmam que a grande maioria das aplicações web não atende aos critérios básicos de conformidade. Como consequência direta, perpetua-se um ambiente digital não inclusivo, que impõe barreiras significativas a pessoas com deficiência.

A baixa disponibilidade de ferramentas específicas que integrem os requisitos de acessibilidade já na fase de prototipação contribui de forma significativa para esse cenário, dificultando a transição entre design e desenvolvimento técnico e comprometendo a entrega de produtos acessíveis.

Diante dessa lacuna, o presente trabalho propôs o desenvolvimento do \textit{Wilia}, um UI Kit para documentação de acessibilidade desenvolvido e  utilizado no Figma, com foco específico na experiência de usuários que utilizam leitores de tela. O kit foi concebido como uma solução prática e tecnicamente embasada, capaz de guiar designers na criação de protótipos acessíveis e de fornecer aos desenvolvedores a documentação necessária para a implementação correta dos elementos. A estrutura do kit foi baseada no conceito de Atomic Design, nas diretrizes da WCAG 2.2 e da WAI-ARIA, bem como nas más práticas apontadas por estudos como os da WebAIM, HandTalk e o Movimento Web Para Todos, permitindo a criação de um kit de componentes robusto, escalável e funcional.

Os resultados obtidos com o desenvolvimento do kit demonstram a viabilidade da aplicação de normas técnicas de acessibilidade de forma integrada ao processo de design visual, com uma documentação que alinha parte visual com textual, flexível e com isso, facilmente aplicável em diversos projetos.

O kit então organizado em páginas temáticas, como Regiões, Cabeçalhos, Imagens, entre outras, com a criação de etiquetas visuais explicativas e a incorporação de documentação lateral no próprio ambiente do Figma, facilitando o entendimento e a aplicação dos princípios de acessibilidade tanto para designers quanto para desenvolvedores. 

Ao traduzir padrões técnicos, que muitas vezes podem ser complexos, em componentes visuais interativos, o \textit{Wilia} atua como uma conexão entre teoria e prática, alinhando-se as diretrizes do Plano Nacional de Direitos da Pessoa com Deficiência ao incentivar o uso de ferramentas que democratizam a acessibilidade digital desde as fases iniciais do projeto.

Mesmo que este kit, por enquanto, foque apenas em leitores de tela, ele foi feito pensando em crescer. Essa limitação atual já mostra o que podemos fazer no futuro.

Como próximos passos, o mais importante é validar o Wilia na prática. Queremos fazer uma pesquisa com designers e desenvolvedores para entender como eles usam o kit e medir se ele realmente ajuda a criar projetos mais acessíveis. O feedback deles será usado para melhorar o material.

Além disso, o plano é expandir o kit. Primeiro, queremos adicionar especificações para outras deficiências, como regras para acessibilidade motora ou cognitiva. Outra ideia é fazer a tradução do \textit{Wilia} para outras línguas, como inglês e espanhol, para que mais usuários possam usar.

Por fim, uma ideia mais complexa para o futuro seria a criação de um plugin (extensão) do \textit{Wilia} para o Figma. Um plugin poderia automatizar parte da documentação ou sugerir atributos de acessibilidade, o que seria um grande avanço para a ferramenta.


%-----------------------------------
% Finaliza a parte no bookmark do PDF
% para que se inicie o bookmark na raiz
% e adiciona espaço de parte no Sumário
\phantompart

%-----------------------------------
% Bibliografia
\bibliography{referencias.bib}


%-----------------------------------
% POST-TEXTUAL ELEMENTS
%-----------------------------------
\postextual

% Anexos
%-----------------------------------
\begin{anexosenv}
\label{anexos}
%-----------------------------------

\chapter{Visão por componente Wilia} \label{anexoA}

\section{Regiões}
\begin{center}
  \captionof{figure}{Componente: Regiões/Landmarks}
  \label{fig:componente-regioes-landmarks}
  \vspace{0.6\baselineskip}    \includegraphics[width=\textwidth,height=0.85\textheight,keepaspectratio]{Figuras/Componente-regioes.png}
  \par\small Fonte: Elaborado pelo autor, 2025.
\end{center}
\clearpage

\section{Cabeçalhos}
\begin{center}
  \captionof{figure}{Componente: Cabeçalhos}
  \label{fig:componente-cabecalhos}
  \vspace{0.6\baselineskip}  \includegraphics[width=\textwidth,height=0.85\textheight,keepaspectratio]{Figuras/Componente-cabecalho.png}
  \par\small Fonte: Elaborado pelo autor, 2025.
\end{center}
\clearpage

\section{Botões}
\begin{center}
  \captionof{figure}{Componente: Botões}
  \label{fig:componente-botoes}
  \vspace{0.6\baselineskip}  \includegraphics[width=\textwidth,height=0.85\textheight,keepaspectratio]{Figuras/Componente-botoes.png}
  \par\small Fonte: Elaborado pelo autor, 2025.
\end{center}
\clearpage

\section{Campos de Entrada}
\begin{center}
  \captionof{figure}{Componente: Campos de Entrada}
  \label{fig:componente-campos-entrada}
  \vspace{0.6\baselineskip}  \includegraphics[width=\textwidth,height=0.85\textheight,keepaspectratio]{Figuras/Componente-campos-de-entrada.png}
  \par\small Fonte: Elaborado pelo autor, 2025.
\end{center}
\clearpage

\section{Imagens}
\begin{center}
  \captionof{figure}{Componente: Imagens}
  \label{fig:componente-imagens}
  \vspace{0.6\baselineskip}    \includegraphics[width=\textwidth,height=0.85\textheight,keepaspectratio]{Figuras/Componente-imagens.png}
  \par\small Fonte: Elaborado pelo autor, 2025.
\end{center}
\clearpage

\section{Links}
\begin{center}
  \captionof{figure}{Componente: Links}
  label{fig:componente-links}
  \vspace{0.6\baselineskip}    \includegraphics[width=\textwidth,height=0.85\textheight,keepaspectratio]{Figuras/Componente-hiperlinks.png}
  \par\small Fonte: Elaborado pelo autor, 2025.
\end{center}
\clearpage

\section{Ordem de Leitura}
\begin{center}
  \captionof{figure}{Componente: Ordem de Leitura}
  \label{fig:componente-ordem-leitura}
  \vspace{0.6\baselineskip}    \includegraphics[width=\textwidth,height=0.85\textheight,keepaspectratio]{Figuras/Componente-ordem-de-leitura.png}
  \par\small Fonte: Elaborado pelo autor, 2025.
\end{center}
\clearpage

\section{Ordem de Foco}
\begin{center}
  \captionof{figure}{Componente: Ordem de Foco}
  \label{fig:componente-ordem-foco}
  \vspace{0.6\baselineskip}    \includegraphics[width=\textwidth,height=0.85\textheight,keepaspectratio]{Figuras/Componente-ordem-de-foco.png}
  \par\small Fonte: Elaborado pelo autor, 2025.
\end{center}

\end{anexosenv}


% Apêndice
%\include{pos-textuais/apendice}
\usepackage{placeins}

\end{document}
