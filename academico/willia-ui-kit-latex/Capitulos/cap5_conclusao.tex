\chapter{Conclusão} 
\label{cap5_conclusao} 

A implementação da acessibilidade digital é, simultaneamente, uma obrigação ética, legal e técnica, e um dos maiores desafios para as áreas de design e desenvolvimento web no cenário global. A prática corrente demonstra uma desconexão preocupante entre a existência de diretrizes internacionais e sua efetiva implementação. Análises quantitativas de larga escala confirmam que a grande maioria das aplicações web não atende aos critérios básicos de conformidade. Como consequência direta, perpetua-se um ambiente digital não inclusivo, que impõe barreiras significativas a pessoas com deficiência.

A baixa disponibilidade de ferramentas específicas que integrem os requisitos de acessibilidade já na fase de prototipação contribui de forma significativa para esse cenário, dificultando a transição entre design e desenvolvimento técnico e comprometendo a entrega de produtos acessíveis.

Diante dessa lacuna, o presente trabalho propôs o desenvolvimento do \textit{Wilia}, um UI Kit para documentação de acessibilidade desenvolvido e  utilizado no Figma, com foco específico na experiência de usuários que utilizam leitores de tela. O kit foi concebido como uma solução prática e tecnicamente embasada, capaz de guiar designers na criação de protótipos acessíveis e de fornecer aos desenvolvedores a documentação necessária para a implementação correta dos elementos. A estrutura do kit foi baseada no conceito de Atomic Design, nas diretrizes da WCAG 2.2 e da WAI-ARIA, bem como nas más práticas apontadas por estudos como os da WebAIM, HandTalk e o Movimento Web Para Todos, permitindo a criação de um kit de componentes robusto, escalável e funcional.

Os resultados obtidos com o desenvolvimento do kit demonstram a viabilidade da aplicação de normas técnicas de acessibilidade de forma integrada ao processo de design visual, com uma documentação que alinha parte visual com textual, flexível e com isso, facilmente aplicável em diversos projetos.

O kit então organizado em páginas temáticas, como Regiões, Cabeçalhos, Imagens, entre outras, com a criação de etiquetas visuais explicativas e a incorporação de documentação lateral no próprio ambiente do Figma, facilitando o entendimento e a aplicação dos princípios de acessibilidade tanto para designers quanto para desenvolvedores. 

Ao traduzir padrões técnicos, que muitas vezes podem ser complexos, em componentes visuais interativos, o \textit{Wilia} atua como uma conexão entre teoria e prática, alinhando-se as diretrizes do Plano Nacional de Direitos da Pessoa com Deficiência ao incentivar o uso de ferramentas que democratizam a acessibilidade digital desde as fases iniciais do projeto.

Mesmo que este kit, por enquanto, foque apenas em leitores de tela, ele foi feito pensando em crescer. Essa limitação atual já mostra o que podemos fazer no futuro.

Como próximos passos, o mais importante é validar o Wilia na prática. Queremos fazer uma pesquisa com designers e desenvolvedores para entender como eles usam o kit e medir se ele realmente ajuda a criar projetos mais acessíveis. O feedback deles será usado para melhorar o material.

Além disso, o plano é expandir o kit. Primeiro, queremos adicionar especificações para outras deficiências, como regras para acessibilidade motora ou cognitiva. Outra ideia é fazer a tradução do \textit{Wilia} para outras línguas, como inglês e espanhol, para que mais usuários possam usar.

Por fim, uma ideia mais complexa para o futuro seria a criação de um plugin (extensão) do \textit{Wilia} para o Figma. Um plugin poderia automatizar parte da documentação ou sugerir atributos de acessibilidade, o que seria um grande avanço para a ferramenta.
