\chapter{Introdução} 
\label{cap1_introducao} 



A acessibilidade digital é um princípio ético e legal fundamental para garantir a inclusão de pessoas com deficiência no uso de tecnologias. No Brasil, 18,6 milhões de pessoas vivem com algum tipo de deficiência, representando 8,9\% da população. Desse percentual, 3,1\% são pessoas de 2 anos ou mais de idade com dificuldade para enxergar, mesmo usando óculos ou lentes de contato. Esse percentual fica somente atrás dos 3,7\% que representam pessoas com dificuldade para andar ou subir degraus \cite{ibge_2022}. 

Uma pesquisa realizada pela \citeonline{bigdatacorp_2024} em parceria com o Movimento Web para Todos \cite{mwpt_2024}, estampou que somente 2,9\%  dos sites brasileiros passaram em todos os testes de acessibilidade. Além disso, somente 21\% das pessoas com deficiência acreditam que os sites e aplicativos atendem suas necessidades, refletindo uma falta significativa na inclusão digital no país \cite{handtalk2023}. Segundo \citeonline{handtalk2023}, 72\% dos componentes de interface em sistemas vão em desencontro com as diretrizes do W3C que define como tornar conteúdos e interfaces webs dinâmicas acessíveis para pessoas com deficiência, para dinamicidade. Com base nisso, pode-se questionar, qual a dificuldade em manter sites e sistemas acessíveis para todos? \begin{comment}
sendo algumas hipóteses levantadas sobre a falta de ferramentas para mitigar esses problemas e também um possível déficit formativo dos profissionais da área. A falta de acessibilidade digital, pode não somente, limitar o acesso à informação, mas também impedir que pessoas com deficiência participem plenamente do ambiente digital.\end{comment}




%Esse déficit formativo explica por que o72\% dos componentes de interface em sistemas descumprem os padrões  WAI-ARIA para dinamicidade. A falta de acessibilidade digital não somente limita o acesso à informação, mas também impede que pessoas com deficiência participem plenamente do ambiente digital. 

%No cenário mundial, o relatório de 2024 sobre a acessibilidade das 1.000.000 principais páginas online \cite{webaim_million_2023}, evidenciou que 56.791.260 erros de acessibilidade distintos foram detectados, uma média de 56,8 erros por página. O número de erros detectados aumentou notavelmente (13,6\%) desde a análise de 2023, que encontrou 50 erros/página. “Erros” são barreiras de acessibilidade com impacto notável no usuário final, mostrando uma probabilidade muito alta de serem falhas de conformidade com WCAG 2.2.

% Esses cenários evidenciam que a maioria das empresas ainda enfrenta desafios na implementação efetiva das diretrizes WCAG e WAI-ARIA. Há uma notável desconexão entre design e implementação técnica, manifestada nos números negativos de sistemas e páginas onlines repercutidos nas pesquisas analisadas.
 
Outro fator que pode contribuir para a intensificação do problema é a possível ausência de ferramentas padronizadas destinadas à especificação de requisitos de acessibilidade durante a etapa de prototipação. Visto que, existe uma dificuldade de se encontrar pesquisas e ferramentas na área. Essa lacuna pode dificultar a comunicação entre designers e desenvolvedores, resultando em entregas imprecisas e aumentando a probabilidade de erros na implementação de funcionalidades acessíveis. O “Estudo sobre o Panorama da Acessibilidade
Digital no Brasil” realizado pela \citeonline{handtalk2023}, diz que menos de 1\% dos sites nacionais analisados possuem mecanismos básicos de acessibilidade, comprometendo a experiência de usuários que dependem de tecnologias assistivas e coloca em risco o cumprimento da legislação, como o Estatuto da Pessoa com Deficiência.


%A ausência de ferramentas padronizadas e eficientes que possibilitem a especificação de requisitos de acessibilidade em protótipos gera dificuldades significativas na comunicação entre designers e desenvolvedores. Essa lacuna resulta em entregas imprecisas, aumentando o risco de erros na implementação de funcionalidades acessíveis.  menos de 1\% dos sites avaliados no Brasil não atendem aos critérios mínimos de acessibilidade, impactando diretamente a experiência de usuários que utilizam tecnologias assistivas e mesmo conformidade legal como o Estatuto da Pessoa com Deficiência.





%As Diretrizes de Acessibilidade para Conteúdo Web (WCAG) estabelecem claramente que interfaces digitais devem ser perceptíveis, operáveis, compreensíveis e robustas, a fim de atender às necessidades de pessoas com diferentes tipos de deficiência \cite{w3c_diretrizes}. No entanto, as empresas enfrentam desafios na implementação efetiva dessas diretrizes, assim como das especificações WAI-ARIA. 

%Essa lacuna torna-se ainda mais crítica ao observarmos que 61\% das empresas brasileiras não oferecem treinamento sobre práticas acessíveis para suas equipes de desenvolvimento, resultando em componentes digitais que frequentemente descumprem os padrões definidos pelo Consórcio World Wide Web (W3C), conforme o estudo “O Panorama da Acessibilidade Digital” realizado pela \cite{handtalk_panorama_2023}. O estudo também aponta que, embora os profissionais de design demonstrem maior sensibilidade ao tema da acessibilidade, os desenvolvedores enfrentam dificuldades concretas para transformar os protótipos acessíveis em soluções técnicas adequadas, devido à falta de capacitação, documentação clara e integração entre áreas.Essa constatação é reforçada por estudo da WebAIM, que identificou que muitos desenvolvedores não recebem instruções claras sobre como aplicar acessibilidade técnica em seus projetos, e frequentemente não têm acesso a ferramentas que facilitem a validação de componentes acessíveis durante o desenvolvimento \cite{webaim_survey_2023}.Ao considerar o desenvolvimento de sistemas acessíveis como foco, é possível identificar uma lacuna significativa entre a etapa de design e a de desenvolvimento. A ausência de ferramentas práticas que promovam uma transição fluida entre o protótipo e o código, especialmente com foco em leitores de tela — recurso essencial para pessoas com deficiência visual —, contribui para a persistência de produtos digitais excludentes.

Neste contexto, UI Kits (\textit{User Interface Kits}), são conjuntos prontos de elementos gráficos e componentes de interface que facilitam o design e o desenvolvimento visual de aplicações, sites e sistemas. Essas ferramentas são amplamente usadas por designers de interface e desenvolvedores de interface para acelerar o processo de criação de produtos digitais. O destaque para o problema da acessibilidade é que elas contêm documentações inclusivas, representando um avanço estratégico no desenvolvimento web.  Segundo a \textit{Interaction Design Foundation}, esses kits colaboram para a consistência, agilidade de prototipagem e alinhamento entre design e código-fonte da aplicação, incluindo elementos essenciais para uma interface acessível por incorporarem componentes pré-configurados com contraste adequado, navegação por teclado e rotulagem semântica \cite{idf_ui_kits}. %Já o framework UIkit (\textit{User Interface Kits}), que são conjuntos de elementos visuais prontos (como botões, menus, formulários e ícones), tem como objetivo auxiliar designers e desenvolvedores a criar interfaces de forma mais rápida, padronizada e consistente.
%Eles demonstram como kits acessíveis podem oferecer componentes interativos que já seguem boas práticas de usabilidade e WAI-ARIA.
%A WAI-ARIA (\textit{Web Accessibility Initiative – Accessible Rich Internet Applications}) é uma especificação do W3C que define atributos para tornar conteúdos e aplicações web mais acessíveis a pessoas com deficiência, especialmente usuários de tecnologias assistivas, como leitores de tela.
Esses kits servem como exemplos técnicos eficazes para designers e desenvolvedores \cite{uikit_accessibility}.

%Integrar a acessibilidade desde o início é mais eficiente e econômico do que corrigir posteriormente, pois evita custos extras e retrabalho, conforme \cite{montana_accessibility}. Além disso, construir sites inclusivos deve abranger múltiplos tipos de deficiência (como visão e habilidade motora), e que recursos acessíveis integrados ao UI Kit facilitam essa aplicação já na fase inicial \cite{webdev_accessibility}.

%Dessa maneira, UI Kits acessíveis funcionam como uma ponte técnica entre designers e desenvolvedores por padronizarem componentes, documentarem requisitos de acessibilidade (como WAI‑ARIA) e reduzirem incertezas no \textit{hand-off}, promovendo sistemas web verdadeiramente inclusivos sem prejudicar a produtividade. A proposta deste trabalho alinha-se diretamente a essa necessidade, ao tornar simples e prática a documentação técnica para leitores de tela desde o design inicial.

%Em pesquisa conduzida pela \citeonline{nttdata}, foram analisadas as preferências de usuários de leitores de tela no Brasil, com base em 564 respostas válidas. Observou-se que, no desktop, a maioria utiliza o NVDA em conjunto com os navegadores Chrome ou Firefox. Já em dispositivos móveis, os leitores mais usados são o TalkBack com Chrome, o VoiceOver com Safari e o JieShuo (chinês) com Chrome. Esses dados ajudam a direcionar o desenvolvimento para os navegadores e ferramentas mais utilizados pelo público.

Assim, o presente trabalho apresenta o \textit{\textbf{Wilia}},  um Kit de Interface de Usuário para documentar especificações de acessibilidade em projetos WEB. Com o foco nos designers, a ferramenta visa promover recursos que promovem protótipos com documentação clara e tecnicamente embasada, pensando nos requisitos de acessibilidade. O presente trabalho tem foco exclusivo na acessibilidade via leitores de tela.
%A acessibilidade digital é uma exigência legal no Brasil (Lei nº 13.146/2015 - LBI) e internacional (ADA, EUA; EAA, União Europeia). No entanto, dados da pesquisa "O Panorama da Acessibilidade Digital" \cite{handtalk_panorama_2023} revelam que cerca de 61\% das empresas brasileiras não oferecem treinamento sobre práticas acessíveis e somente 21\% das pessoas com deficiência acreditam que sites e aplicativos atendem de fato suas necessidades.
%A nível global, o relatório anual WebAIM Million (2025) constatou que 94,8\% das páginas iniciais avaliadas apresentavam falhas em acessibilidade automaticamente detectáveis, com uma média de 56,8 erros por página, sendo que os mais comuns envolviam contraste insuficiente, ausência de texto alternativo e má estrutura semântica \cite{webaim_million_2025}.
%O projeto alinha-se ao Plano Nacional de Direitos da Pessoa com Deficiência \cite{pndpd_2023}, que prioriza a criação de ferramentas para democratizar a implementação de padrões acessíveis em todas as fases do desenvolvimento digital. 
Ao traduzir normas técnicas em componentes práticos, o UI Kit proposto serve como ponte entre a teoria da acessibilidade e sua aplicação concreta, contribuindo para a redução da dependência de desenvolvedores em relação a documentações vagas e a garantir a conformidade legal e ética, melhorando a experiência de usuários com deficiência.

\section{Objetivos}

\subsection{Objetivo Geral} 
Criar um UI Kit para Figma que permita documentar especificações de acessibilidade em protótipos, que busca mitigar erros de implementação semântica, descritiva e operacional. 
%observados em análises da \cite{webaim_million_2023} especificamente para leitores de tela.

\subsection{Objetivos Específicos} 
\begin{itemize}
\item  Levantar e sintetizar as diretrizes de acessibilidade da WCAG (\textit{Web Content Accessibility Guidelines}) e WAI-ARIA (\textit{Web Accessibility Initiative – Accessible Rich Internet Applications}).
\item Desenvolver um Kit de Interface de Usuário (UI Kit) na plataforma Figma.
\item Mapear componentes para documentar acessibilidade para protótipos HTML.
\item Criar material educativo: Guia prático para designers sobre a aplicação dos componentes e sua relação com requisitos técnicos.
\item Proporcionar um método claro e integrado ao fluxo de design 
%\begin{itemize}
%\item \textbf{Textos Alternativos (Alt Text)}: Componentes para rotular imagens e ícones, seguindo as recomendações da WCAG 1.1.
%\item \textbf{Hierarquia de Títulos}: Componentes para estruturar cabeçalhos (H1 a H6) e garantir navegação semântica.
%\item \textbf{Foco de Teclado}: Componentes para especificar a ordem e comportamento do foco durante a navegação por teclado (WCAG 2.4.3).
%\item \textbf{Rótulos de Formulário}: Componentes para associar rótulos a campos de entrada, evitando erros como "campo sem descrição" (WCAG 3.3.2).
%\item \textbf{Atributos ARIA}: Componentes para adicionar atributos como aria-label e aria-live em elementos interativos.
%\item \textbf{Estrutura Semântica}: Componentes para organizar conteúdo com landmarks (ex.: <nav>, <footer>) e listas.
%\end{itemize}
%\item Disponibilizar exemplos e variações de componentes: Baseados em padrões internacionais (W3C) e leis (LBI, ADA).
%\item Demonstrar a redução de erros: Comparação entre protótipos com e sem o uso do kit, focando em problemas comuns (ex.: 72% dos componentes governamentais descumprindo WAI-ARIA, conforme \cite{webaim_million_2023}).
\end{itemize}



%\section{Justificativa/Relevância}




% A acessibilidade é uma exigência legal em muitos países, incluindo o Brasil, que dispõe da Lei Nº 13.146, de 6 de julho de 2015 Lei Brasileira de Inclusão \cite{lbi_brasil_2015}). Temos as regulamentações internacionais como a ADA, Americans with Disabilities Act \cite{ada_usa_2009}, nos EUA e a EAA, European Accessibility Act \cite{eaa_europa_2019}, na União Europeia. No âmbito ético, a acessibilidade é essencial para garantir a dignidade e autonomia de usuários com deficiências. No entanto, a falta de ferramentas práticas para designers contribui para a baixa adesão às normas \cite{handtalk_panorama_2023}. Este projeto visa preencher essa lacuna, oferecendo uma solução teórica, prática e alinhada a padrões globais.

% O projeto também se alinha às recomendações do Plano Nacional de Direitos da Pessoa com Deficiência \cite{pndpd_2023}, que prioriza a criação de ferramentas para "democratizar a implementação de padrões acessíveis em todas as fases do desenvolvimento digital". Ao traduzir normas técnicas em componentes práticos, o UI Kit proposto serve como ponte entre a teoria da acessibilidade e sua aplicação concreta no ecossistema digital.



%\section{Métodos}



 %A metodologia adotada inclui as seguintes etapas:

 %\begin{itemize}
 %    \item Pesquisa Bibliográfica: Análise das diretrizes internacionais de acessibilidade (WCAG, WAI-ARIA, LBI, ADA e EAA), identificação dos principais desafios enfrentados por designers e desenvolvedores e analise sobre a ótica de tecnologias assistivas.
  %   \item Análise de Ferramentas Existentes: Avaliação de soluções disponíveis no mercado para identificar lacunas e oportunidades.
   %  \item Desenvolvimento do UI Kit: Criação dos componentes no Figma, com base nas diretrizes identificadas e de modo a possibilitar uma especificação que vise mitigar os principais problemas apontados pela \cite{webaim_million_2023} nos sites.
    % \item Validação: Alinhamento dos componentes com as normas internacionais e disponibilização de exemplos práticos.
    %\item Documentação: Elaboração de materiais educativos que orientem o uso correto dos componentes.

 %\end{itemize}

%%%%%%%%%%%%
%Faça nesse estilo: 

%\section{ORGANIZAÇÃO DO TEXTO}
%Este trabalho está estruturado em cinco capítulos. O presente capítulo, \textbf{Introdução}, apresenta o tema do estudo, define os objetivos gerais e específicos e descreve a organização do texto. O segundo capítulo, \textbf{Revisão da Literatura}, discute trabalhos relacionados ao tema, com destaque para a pesquisa de \citeonline{silva2024} e para o processo de visualização de dados. Também são explorados, nesse capítulo, aspectos do desenvolvimento web, tanto no front-end quanto no back-end, além da apresentação de dois exemplos de plataformas web voltadas à visualização de dados.

%O terceiro capítulo, \textbf{Metodologia}, detalha todo o processo de visualização de dados e a forma como ele foi aplicado. Abrange ainda a modelagem e criação do banco de dados, bem como o desenvolvimento das partes front-end e back-end da plataforma. O quarto capítulo, \textbf{Resultados}, apresenta os resultados obtidos, incluindo as representações gráficas dos dados coletados.

%Por fim, o quinto capítulo, \textbf{Conclusão}, traz as considerações finais sobre a plataforma desenvolvida, reflexões sobre os resultados e sugestões para trabalhos futuros.

\section{Estrutura dos capítulos}
Este trabalho está estruturado em cinco capítulos. O presente capítulo, \textbf{Introdução}, apresenta o tema do estudo, define os objetivos gerais e específicos e descreve a organização do texto. O segundo capítulo, \textbf{Referencial Teórico}, discute trabalhos relacionados ao tema e apresenta com maior profundidade os temas que norteiam essa pesquisa como acessibilidade digital, tecnologias assistivas, WCAG e WAI-ARIA, além da legislação a respeito e a ferramenta Figma.

O terceiro capítulo, \textbf{Metodologia}, detalha todo o processo de construção da pesquisa, incluindo a etapa de prototipação e organização no Figma. O quarto capítulo, \textbf{Resultados}, apresenta os resultados obtidos, incluindo o mapeamento dos aspectos gráficos no Figma e criação do material educativo.

Por fim, o quinto capítulo, \textbf{Conclusão}, traz as considerações finais sobre a pesquisa desenvolvida, reflexões sobre os resultados e sugestões para trabalhos futuros.








% No mundo digital atual, onde tecnologias como sites, aplicações e plataformas online desempenham um papel fundamental na vida cotidiana, garantir que essas interfaces sejam acessíveis a todas as pessoas é uma obrigação ética, social e legal. Globalmente, cerca de 1,3 bilhão de pessoas, ou 16\% da população mundial, vivem com algum tipo de deficiência, segundo a Organização Mundial da Saúde \cite{oms_2021}. No Brasil, esse número chega a mais de 17 milhões de pessoas, quase 8\% da população, conforme apontado por uma pesquisa realizada pela \cite{handtalk_panorama_2023}. Apesar disso, apenas 21\% dessas pessoas acreditam que sites e aplicativos contemplam suas necessidades de navegação, e menos de 1\% dos sites no país contam com serviços de acessibilidade focados em pessoas com deficiência.

% O cenário global reflete uma realidade semelhante. A Web Accessibility in Mind \cite{webaim_million_2023}, uma organização sem fins lucrativos, analisou um milhão de homepages e revelou que 98\% delas apresentam pelo menos um erro de acessibilidade significativo, como problemas de contraste de cores, falta de textos alternativos para imagens e dificuldades de navegação por teclado. Esses números evidenciam um abismo entre o conhecimento sobre acessibilidade digital e sua implementação prática.

% As Diretrizes de Acessibilidade para Conteúdo Web (WCAG), publicadas pelo Consórcio World Wide Web \cite{w3c_diretrizes}, estabelecem claramente que interfaces digitais devem ser perceptíveis, operáveis, compreensíveis e robustas para atender às necessidades de pessoas com diferentes tipos de deficiência. No entanto, muitos designers e desenvolvedores enfrentam dificuldades na implementação prática desses princípios. Um estudo conduzido por \cite{gonzalez2022} identificou que a falta de clareza na especificação de requisitos de acessibilidade durante a fase de design é uma das principais barreiras para garantir interfaces verdadeiramente inclusivas.

% Nesse contexto, surge a necessidade de soluções práticas que facilitem a especificação de requisitos técnicos de acessibilidade desde o início do processo de design. Para resolver essa problemática, este trabalho propõe desenvolver uma Biblioteca de Especificações de Acessibilidade no Figma, uma ferramenta prática voltada para designers que buscam documentar clara e tecnicamente os requisitos de acessibilidade em seus protótipos.

% A biblioteca será estruturada para oferecer aos designers uma série de componentes e guias visuais que permitem marcar e especificar requisitos de acessibilidade de forma compreensível para desenvolvedores. Entre os principais elementos incluídos, destacam-se:

% \begin{itemize}
%     \item Textos Alternativos (Alt Text) : Componentes que explicam como rotular imagens e ícones para leitores de tela.
%     \item Hierarquia de Títulos: Modelos que demonstram como organizar cabeçalhos (H1, H2, etc.) para garantir uma navegação semântica.
%     Contraste de Cores: Paletas de cores pré-definidas que atendem aos níveis de contraste recomendados pela WCAG.
%     \item Foco de Teclado: Exemplos visuais de como o foco deve se comportar durante a navegação por teclado.
%     Rótulos de Formulários: Componentes que mostram como associar rótulos a campos de entrada para garantir acessibilidade.
%     \item Atributos ARIA: Guia prático para o uso correto de atributos ARIA em componentes interativos.
%     Estrutura Semântica: Modelos que demonstram como organizar conteúdo com cabeçalhos, listas e landmarks (ex.: <header>, <main>, <footer>).
% \end{itemize}

% Esses componentes serão baseados nas diretrizes internacionais da WCAG e WAI-ARIA, garantindo conformidade com padrões consolidados e facilitando a comunicação entre equipes multidisciplinares.



% \section{Objetivos}

% Este trabalho pretende desenvolver uma ferramenta prática que auxilie designers na especificação técnica de requisitos de acessibilidade em projetos digitais. Mais especificamente, busca-se:

% \begin{itemize}
%     \item Criar uma biblioteca organizada e intuitiva que forneça componentes e guias visuais para documentar requisitos de acessibilidade.
%     \item Basear a biblioteca nas diretrizes da WCAG 2.2 e WAI-ARIA , promovendo práticas inclusivas no design de interfaces digitais.
%     \item Facilitar a comunicação entre designers e desenvolvedores, reduzindo lacunas na implementação de acessibilidade.
% \end{itemize}



% \section{Justificativa e relevância}

% A criação desta biblioteca é essencial para apoiar e promover a inclusão digital, facilitando a comunicação entre designers e desenvolvedores e tornando o desenvolvimento mais preciso. Além disso, a biblioteca abordará pontos cruciais, como:

% \begin{itemize}
%     \item Impacto Social: Promover a inclusão digital, garantindo que pessoas com deficiência tenham acesso igualitário às tecnologias digitais.
%     \item Conformidade Legal: Atender às exigências legais de acessibilidade digital estabelecidas por leis como a LBI, Lei Brasileira de Inclusão \cite{lbi_brasil_2015}, a ADA, Americans with Disabilities Act \cite{ada_usa_2009}, nos EUA e a EAA, European Accessibility Act \cite{eaa_europa_2019}, na União Europeia.
%     \item Benefícios Empresariais: Interfaces acessíveis melhoram a experiência do usuário, ampliam o alcance do público-alvo e reduzem riscos legais.
%     \item Inovação Tecnológica: Contribui para o avanço do design inclusivo, preenchendo uma lacuna no ecossistema de ferramentas de design digital.
%     \item Conscientização e Aprendizado: Além de facilitar a especificação técnica de requisitos de acessibilidade, a biblioteca pode servir como uma ferramenta educacional poderosa. Ao fornecer exemplos práticos e guias claros baseados nas diretrizes da WCAG e WAI-ARIA, possibilitando designers e desenvolvedores entenderem melhor os princípios fundamentais da acessibilidade digital. Essa abordagem prática promove a conscientização sobre a importância de projetar soluções verdadeiramente inclusivas, contribuindo para um ecossistema digital mais acessível e empático.
% \end{itemize}



% \section{Métodos}

% A metodologia adotada inclui as seguintes etapas:

% \begin{itemize}
%     \item Pesquisa Bibliográfica : Análise de diretrizes da WCAG , WAI-ARIA e outras referências sobre acessibilidade digital.
%     \item Análise de Ferramentas Existentes : Avaliação de soluções disponíveis no mercado para identificar lacunas e oportunidades.
%     \item Prototipagem no Figma : Desenvolvimento da Biblioteca de Especificações com base nas diretrizes pesquisadas.
%     \item Documentação dos Componentes : Criação de guias detalhados para explicar o uso de cada componente.
% Estrutura dos Capítulos
% \end{itemize}



% \section{Estrutura dos Capítulos}

% O trabalho está organizado da seguinte forma:

% \begin{itemize}
%     \item Capítulo 1 - Introdução : Apresentação do tema, objetivos, justificativa, métodos e estrutura do trabalho.
%     \item Capítulo 2 - Referencial Teórico : Fundamentação teórica sobre acessibilidade digital, tecnologias assistivas, WCAG e WAI-ARIA.
%     \item Capítulo 3 - Metodologia : Detalhamento dos métodos utilizados no desenvolvimento da Biblioteca de Especificações.
%     \item Capítulo 4 - Resultados : Apresentação da Biblioteca finalizada e descrição de seus componentes.
%     \item Capítulo 5 - Conclusão : Síntese dos principais aprendizados e sugestões para trabalhos futuros.
%     \item Referências : Lista de todas as fontes consultadas.
% \end{itemize}

% Com essa abordagem, espera-se contribuir significativamente para a democratização do design inclusivo, impactando positivamente a experiência de milhões de usuários que dependem de interfaces acessíveis para navegar na web.






% \subsection{Objetivo Geral} 

% Estruturado no Figma, esse kit de Interface de Usuário será estruturado para oferecer aos designers uma série de componentes e guias visuais que permitem marcar e especificar requisitos de acessibilidade de forma compreensível para desenvolvedores. Entre os principais elementos incluídos, destacam-se:


% \begin{itemize}
%      \item Textos Alternativos (Alt Text) : Componentes que explicam como rotular imagens e ícones para leitores de tela.
%      \item Hierarquia de Títulos: Modelos que demonstram como organizar cabeçalhos (H1, H2, etc.) para garantir uma navegação semântica.
%      Contraste de Cores: Paletas de cores pré-definidas que atendem aos níveis de contraste recomendados pela WCAG.
%      \item Foco de Teclado: Exemplos visuais de como o foco deve se comportar durante a navegação por teclado.
%      Rótulos de Formulários: Componentes que mostram como associar rótulos a campos de entrada para garantir acessibilidade.
%      \item Atributos ARIA: Guia prático para o uso correto de atributos ARIA em componentes interativos.
%      Estrutura Semântica: Modelos que demonstram como organizar conteúdo com cabeçalhos, listas e landmarks (ex.: <header>, <main>, <footer>).
%  \end{itemize}

%  Esses componentes serão baseados nas diretrizes internacionais da \cite{wcag_22_2023}, garantindo conformidade com padrões consolidados e facilitando a comunicação entre equipes multidisciplinares. 



% \section{Justificativa/Relevância}
% Este UI Kit de Acessibilidade será essencial para apoiar e promover a inclusão digital, facilitando a comunicação entre designers e desenvolvedores e tornando o desenvolvimento mais preciso. Além disso, o Kit abordará pontos cruciais, como:

%  \begin{itemize}
%      \item Impacto Social: Promover a inclusão digital, garantindo que pessoas com deficiência tenham acesso igualitário às tecnologias digitais.
%      \item Conformidade Legal: Incentivar que projetos web sejam implementados cumprindo às exigências legais de acessibilidade digital estabelecidas por leis como a LBI, Lei Brasileira de Inclusão \cite{lbi_brasil_2015}, e até mesmo as regulações internacionais como a ADA, Americans with Disabilities Act \cite{ada_usa_2009}, nos EUA e a EAA, European Accessibility Act \cite{eaa_europa_2019}, na União Europeia.
%      \item Benefícios Empresariais: Interfaces acessíveis melhoram a experiência do usuário, ampliam o alcance do público-alvo e reduzem riscos legais.
%      \item Inovação Tecnológica: Contribui para o avanço do design inclusivo, preenchendo uma lacuna no ecossistema de ferramentas de design digital.
%      \item Conscientização e Aprendizado: Além de facilitar a especificação técnica de requisitos de acessibilidade, a biblioteca pode servir como uma ferramenta educacional poderosa. Ao fornecer exemplos práticos e guias claros baseados nas diretrizes da WCAG e WAI-ARIA, possibilitando designers e desenvolvedores entenderem melhor os princípios fundamentais da acessibilidade digital. Essa abordagem prática promove a conscientização sobre a importância de projetar soluções verdadeiramente inclusivas, contribuindo para um ecossistema digital mais acessível e empático.
%  \end{itemize}


% Este trabalho busca contribuir para reduzir essa lacuna ao fornecer uma solução prática e acessível para designers. Ao facilitar a especificação de acessibilidade desde as fases iniciais do design, espera-se que o UI Kit proposto melhore a assertividade na entrega de protótipos, otimize a comunicação entre equipes multidisciplinares e, consequentemente, promova uma experiência digital mais inclusiva.
